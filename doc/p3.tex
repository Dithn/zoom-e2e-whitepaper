\section{Phase III: Transparency Tree}

In the third phase, we will expand the authentication guarantees to ensure that all Zoom users have a consistent view of each others' devices and keys.

Imagine an insider, Mallory, who wants to eavesdrop on a meeting between honest users Alice and Bob, who have never interacted on Zoom before and haven't checked the meeting leader security code. To succeed in this attack, Mallory could instruct the Zoom server to lie to Alice about Bob's keys and to Bob about Alice's keys, replacing them with keys she controls. If Bob's client is the only one to see the fake key for Alice, and similarly Alice's is the only client who gets the fake key for Bob, then such an attack would be hard to detect after the fact.

Some possible countermeasures for such attacks require trusted external entities or manual validation steps (such as checking security codes, as introduced in Phase I) that potentially have to be performed out-of-band. Instead, our plan detects equivocation by the Zoom servers and identity providers while minimizing active checking by the user.

In Phase III, we will force the Zoom server to provide the same mapping between user accounts and public keys to all clients, to sign such a mapping, and to be held accountable for these signed statements. This way, in order to compromise a single meeting, Zoom would have to lie not only to Alice about Bob's keys (and vice versa), but also to every other Zoom user about those keys, including lying to Bob about his own keys. Bob's client can thus easily review the list of his devices and discover any suspicious activity. External auditors can then routinely verify that the server's mapping is consistent over time.

Thus the key fingerprint comparisons from the prior two phases can be demoted in the user experience, to be replaced with targeted security alerts (which we expect never to be triggered). Key security becomes virtually invisible to the user.

\subsection{Zoom Transparency Tree}

The idea that there should be a single and consistent mapping between an identity and its public keys has already been explored successfully to solve similar issues. Most notably, Certificate Transparency~\cite{langley2013certificate} limits the damage that a compromised certificate authority can do by signing fake TLS certificates. It does so by requiring that all the signed x509 certificates have to be submitted to a publicly auditable log before being accepted by browsers. Industry projects such as Key Transparency~\cite{keytransparency} and Keybase~\cite{keybase} (which is now part of Zoom), and academic works such as SEEMless~\cite{chase2019seemless} and CONIKS~\cite{melara2015coniks} have explored applying a similar approach to individual users' identities for messaging applications, with Keybase being the only instance in production use today, as far as we know. However, all the existing solutions in this space that we are aware of do not currently match Zoom's security and privacy requirements while offering usability features like multi-device support.

We will build on prior work to design a new mechanism tailored to Zoom's use cases: the Zoom Transparency Tree (ZTT). The ZTT will be backed by a Merkle tree as used by Keybase, but with privacy-preserving path-lookup features such as in CONIKS. This data structure offers a key-value store interface where key-value pairs, once inserted, cannot be removed or altered. The state of the structure can be summarized by a small commitment, and lookup queries can be accompanied by a short proof that they are consistent with the commitment. Whenever a client is given a signed sigchain statement (as introduced in Phase II) about another user's identity or their keys, this statement will be accompanied by an inclusion proof in the ZTT.

\subsection{Integration Details}

\subsubsection{ZTT Auditing}

The design of the ZTT requires auditing to verify the structure of the tree. Zoom will partner with independent external auditors which will (in a privacy-preserving way) ensure that the append-only property of the ZTT is respected.

Clients will query the auditors to ensure that their view of the ZTT's commitment is consistent with everyone else's. If the client can reach the auditor and detects a fork in the ZTT, they can send the auditor the forked and signed commitments in addition to the warning, so that the auditor can disclose the inconsistency. If Zoom clients cannot reach any of the auditor servers, they will signal a degraded encryption level (as elsewhere in this protocol).

We will publish code so that interested parties can also audit the ZTT.

Additionally, organizations using Zoom will be able to review updates to the ZTT and track their employees' device changes.

\subsubsection{Provisioning}

When provisioning a new device, the client will ensure that the sigchain statement is included in the ZTT by first sending it to the Zoom servers and then querying the ZTT to check that the sigchain update has been included.

\subsubsection{Self-Audit and Refresh}

Periodically, the user's client should ask the server for an updated ZTT commitment, ensure that this commitment is consistent with past data, possibly verify it with external auditors, and review the user's sigchain for any new statements. If new keys are added to the sigchain, the client should ask the user to review the changes. If the user notices an unexpected change, they may be prompted to change their password or talk to their IT department.

\subsubsection{Joining a Meeting or Accepting a Join Request}

Because we only trust keys stored within the ZTT, users in a meeting will verify that each others' public keys are included in the ZTT, before proceeding with key exchange as in Phase I. If the verification fails, the client will fail to join the meeting.

\subsubsection{Contact List Updates}

The contact lists that users accumulate in Phase II are now also stored in the ZTT. The immutability guarantees that the ZTT provides means that Alice can note an update to Bob's identity in the client installed on her phone, and Zoom is obliged to relay that update to her desktop client.

\subsubsection{Multiple Trees}

In this setting, we anticipate the need for multiple ZTTs, one for each organization, and a global tree computed over all the organization-specific trees. This tiered architecture has several advantages: it allows organizations who want independence from each other to keep their data disentangled; it enables sensitivity to local data protection laws, as different trees can have different parameters; it improves scalability; it better mirrors Zoom's current server-side infrastructure; it anticipates a future federated architecture; and it simplifies audits, since auditors can focus on the global tree and their organization's tree.

We anticipate a downside from this approach from our experience at Keybase: sometimes it is necessary to coordinate updates to different parts of the tree to ensure strict ``happens before'' relationships across leaves. For instance, Keybase guarantees that if a user signs on behalf of a team, and then revokes that key, the signature happens provably before the revocation.

The multiple tree architecture might make such constraints impossible if across different independent trees, but we anticipate that slight relaxations of certain security requirements will smooth the way.

\subsection{Areas to Improve in Phase III}

Though an equivocating server will be detected, we rely on the user to validate device additions. Users might be offline or might be ignoring notifications and therefore compromises might not be detected, or only detected after an attack.

The ZTT requires external auditors to provide security guarantees. If the auditors are not honest, or have poor uptime, this can limit the ability to detect server misconduct. We can mitigate this risk by relying on multiple auditors or implement partial auditing by the clients.
