Zoom Phone is a cloud phone system, in which Zoom Phone users are assigned phone numbers and can
call other Zoom Phone users, as well as other telephone landline and mobile numbers. As with Zoom
Meetings using enhanced encryption, all Zoom Phone calls are encrypted in transit between client
endpoints and the server, but not in an end-to-end manner: a passive adversary with Zoom server
access may be able to decrypt the data streams. To increase the guarantees for Zoom Phone users and
reduce reliance on the server, we offer the ability to upgrade an ongoing Zoom Phone call to use
end-to-end encryption.

\subsection{E2EE Zoom Phone Calls}\label{subsec:phone}

The design of E2EE for Zoom Phone has very similar goals, threat models, and limitations as in Zoom
Meetings. We wish to prevent even insiders with access to Zoom servers from compromising the
confidentiality and integrity of E2E-encrypted Zoom Phone calls. Like Zoom Meetings, Zoom Phone has
many features that are not compatible with strong cryptographic guarantees, such as dialing in from
PSTN phones. Such features will be disabled when E2E is enabled.

At the moment, we only support E2EE for calls between up to two Zoom Phone clients in the same
account, which greatly simplifies the protocol.

\subsubsection{Join/Leave Protocol}

E2EE Zoom Phone calls leverage the key management system from Sections~\ref{subsec:multidev} and
\ref{subsec:clientkeys}, and the same cryptographic primitives as Zoom Meetings (described in
Section~\ref{subsec:cryptoalgs}). Zoom Phone calls are identified with {\sf CallSessionID} instead
of $\meetingID$ and $\meetingUUID$.

To join an E2E call, participants ask for signed statements from the Zoom server over their ID and
long-term key, just as in Zoom meetings (described in Section~\ref{subsubsec:servercert}). They also
generate per-call ephemeral DH keys $(\pk_i, \sk_i)$, sign them with their device keys (similar to
Section~\ref{subsubsec:partikeygen}), and send the key and signature to the other participant.

The call is encrypted with a shared meeting key obtained from the DH key exchange of the clients'
ephemeral keys (using the participants' UIDs and the {\sf CallSessionID} as context). Since the set
of participants is fixed, the key does not rotate and does not have a sequence number, and there is
also no Leader Participant List, heartbeats, or any analogue of the ``locked meeting'' feature.
Participants still fetch each other's long-term public keys from the key server. Instead of the
bulletin board, the server simply offers an interface for the two callers to message each other. At
the moment, the only way to have an E2EE Zoom Phone call is to upgrade an in-progress call to use
E2EE. When upgrading to E2EE, each client first sends its own signed ephemeral key, and when it
receives the other party's key material, it responds with an explicit acknowledgement message. Once
it receives the other party's acknowledgement, the client can start encrypting its own call stream.

\subsubsection{Phone Security Code}

To defend against MitM attacks, Zoom Phone provides a ``phone security code'' that has a similar
format to the meeting leader security code (Section~\ref{subsec:securitycode}), but that is derived
from the ephemeral public keys of both parties. Since the set of participants is fixed, there is no
concern about this code changing too frequently. The security code is computed as
$$\mathsf{Digits}(\SHATWO(\context || \pk_{\sf Caller} || \pk_{\sf Callee} || {\sf
CallSessionID})),$$ % 
where $\context$ is the string \texttt{"Zoombase-2-ClientOnly-KDF-PhoneSecurityCode"}. The user who
initiated the call is designated the {\sf Caller}, and the other user is the {\sf Callee}.

\subsection{Advanced Encryption for Voicemail}\label{subsec:voicemail} Zoom Phone users may receive
voicemail sent from either mobile and landline phones using PSTN, or other Zoom Phone clients. By
default, voicemail messages are received and recorded by Zoom servers, who encrypt them at rest with
keys controlled by the Zoom infrastructure. 

We also offer Advanced Encryption for Voicemail (AEV) as a more secure alternative: Zoom servers
still receive and record the voicemail themselves, but they encrypt it for keys known only to the
intended recipient's devices. We accomplish this using a voicemail-specific PUK subkey, as described
in Section~\ref{subsubsec:puks}. Accordingly, AEV-encrypted voicemails can only be accessed by the
devices belonging to the intended recipient with access to the PUK, i.e.\ devices created before the
voicemail was generated, or (transitively) approved by such a device. We do not support retrieving
these voicemail messages from PSTN, the Zoom website, or generic desk phone/SIP devices. 

Voicemails encrypted with AEV use a simplified version of the email encryption scheme
(Section~\ref{subsec:emailprotocol}). Voicemail ciphertexts include the audio recording as the only
encrypted piece, and there is always only a single (non-BCC) recipient. Email addresses are omitted
as they are not relevant in the voicemail context, and there are no signatures as, even when sent
among Zoom Phone users, it is the Zoom servers that are performing the recording and encryption.

\subsubsection{Security Properties}
The security properties of AEV largely follow from those of the key management layer that it is
built upon, as detailed in Section~\ref{sec:IdKmProps}. In particular, a malicious insider cannot
decrypt any AEV-encrypted voicemail messages recorded before its compromise began (unless it also
corrupts a device to obtain the necessary decryption keys). Key rotations happen as soon as possible
when the set of devices changes and ensure that revoked devices cannot decrypt future voicemails,
while we rely on the server to withhold past ciphertexts.

We stress that, even if Zoom Phone servers delete plaintext voicemail messages as soon as possible
after encryption (and cannot perform decryption afterwards), they have temporary access to the
voicemails during recording, and therefore AEV has inherently weaker guarantees than end-to-end
encryption. In addition to the confidentiality limitations, voicemail recipients cannot
independently verify, and therefore rely on the Zoom servers for, the integrity of the voicemail
recording (i.e.\ that it wasn't tampered with) and of any metadata such as the recording time,
claimed author, intended recipient and their respective phone numbers. This is unavoidable at least
in the PSTN case, as this technology does not support encryption natively. We are considering
offering E2EE voicemail between Zoom Phone users in the future. Moreover, we note that the above
metadata is not encrypted with per-user keys, and remains available to the server to provide the
service even after the audio recording is not.

Since Zoom servers are performing the encryption and keeping track of everyone's keys, fingerprints
are not displayed when sending or listening to an AEV voicemail message (but the user interface
distinguishes between messages using standard and AEV encryption). Fingerprint verification is still
useful during device approvals to prevent an attacker (including a recently compromised server) from
gaining access to previously encrypted voicemails.
