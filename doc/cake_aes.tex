\section{Cake-AES}\label{sec:cake}

Cake-AES is a general-purpose symmetric encryption mode built on top of 256-bit AES-GCM and SHA-512
with the following features:

\begin{itemize}
\item Support for incremental encryption of long messages (theoretically, up to $2^{109}$ bytes)
\item Support for incremental, authenticated, random-access decryption of long messages
\item A single long-lived key can encrypt a very high number of ciphertexts, by using large 24-byte
nonces
\item Nonces are internally generated, reducing the likelihood of nonce misuse
\item Ciphertexts commit to both the key and (indirectly) the message
\end{itemize}

By contrast, AES-GCM and ChaCha20-Poly1305 both use 12-byte nonces, and have a message size limit of
64 GiB and 256 GiB respectively. While both support random-access decryption, the full message must
be decrypted to be authenticated.

Cake-AES private keys are 32-byte strings generated uniformly at random. At a high level, Cake-AES
derives a one-time key by hashing the private key with the nonce, breaks the plaintext into 16 KiB
chunks, and encrypts each chunk with AES-GCM and a position-based nonce.  Below, we describe in more
depth how to encrypt and decrypt data using Cake-AES.

\subsection{Encryption}

\begin{itemize}
\item Generate a random 24-byte nonce.
\item Concatenate the caller's 32-byte key and the random nonce and hash them with SHA-512. Use the
first 32 bytes of the hash as a one-time key, and the second 32 bytes as a key commitment.
\item Split the message into 16 KiB chunks, with a short final chunk that may be empty.
\item Encrypt each chunk with AES-GCM using the one-time key. For the AES-GCM nonce, use the
little-endian encoding of either twice the chunk index (for non-final chunks) or twice the chunk
index plus one (for the final chunk).
\item Concatenate the random nonce, the key commitment, and all the encrypted chunks to produce the
ciphertext.
\end{itemize}

\subsection{Decryption}

\begin{itemize}
\item Read the random nonce and the key commitment from the front of the ciphertext.
\item Hash the caller's 32-byte key and the random nonce with SHA-512. The first 32 bytes of the
hash are the one-time key. Assert that the second 32 bytes match the key commitment.
\item Split the remainder of the ciphertext into 16400-byte chunks (16 KiB plus 16 bytes for the GCM
tag), with a short final chunk.
\item Decrypt each chunk with AES-GCM using the one-time key and the same chunk nonce schedule as
above.
\item Concatenate all the plaintext chunks to form the original message. Or, since each chunk is
independently authenticated, return chunks of plaintext to the caller as a stream.
\end{itemize}

\subsection{Random-Access Decryption}

\begin{itemize}
\item If the one-time key is not yet cached:
\begin{itemize}
\item Read the random nonce and the key commitment from the front of the stream, derive the one-time
key, and verify the key commitment as above.
\item Cache the one-time key.
\end{itemize}
\item If the authentic plaintext length is not yet cached:
\begin{itemize}
\item Get the apparent ciphertext length from the underlying stream. For example, with a ciphertext
saved on disk, this would be the file size. This value may be inauthentic, and we need to
authenticate it.
\item Compute the apparent length of the final ciphertext chunk. This is the total length, minus the
random nonce and commitment lengths (56), modulo 16400. Subtract that length from the total to give
the apparent start offset of the final ciphertext chunk.
\item Seek the underlying stream to that start offset.
\item Read the final ciphertext chunk and decrypt it with AES-GCM using the one-time key and the
appropriate final chunk nonce.
\item If decryption succeeds, the ciphertext length is authentic. Compute the authentic plaintext
length by subtracting the random nonce and commitment lengths, plus 16 bytes times the number of
chunks (the GCM tags), from the ciphertext length.
\item Cache the authentic plaintext length.
\end{itemize}
\item If the plaintext seek target is beyond the authentic plaintext length, arrange to return EOF
to the caller for any subsequent reads.
\item Otherwise, seek the underlying stream to the start of the ciphertext chunk corresponding to
the caller's plaintext seek target. To account for the seek offset within that chunk, arrange to
skip the appropriate number of bytes after the next decryption.
\end{itemize}
