\textbf{Note:} As of version 5.13.10, we only support Identity Provider
attestations that are issued by Okta and are displayed in the context of E2EE meetings.
This feature is called ``Okta Authentication for End-to-End Encryption''.

Accounts that have an ADN and a compatible identity provider\footnote{Compatible
identity providers need to support a custom extension of the OpenID Connect
(OIDC) protocol which we describe below.} (IDP) are able to have the IDP vouch
for their users' identities in a way that other Zoom users can independently
verify. This mechanism restricts the ability, even for Zoom insiders, to
impersonate account members. Many organizations already trust an IDP for
authentication purposes, so this feature does not increase the attack surface or
require additional trust in the IDP.

In order for clients to be able to verify identity attestations by an external IDP, we need two
components:

\begin{enumerate}
\item A way for clients to determine the IDP associated with a Zoom account (that cannot be
    tampered with by the Zoom servers)
\item A mechanism for IDPs to issue---and for clients to verify---a signed attestation that binds
    a user's email address to the cryptographic key(s) they use to communicate (see Section~\ref{sec:identitykeymanagement})
\end{enumerate}

\subsection{Associating Accounts with Identity Providers}
\label{subsec:adnToIdp}
In order to associate an account to its IDP, the account's ADN hosts a DNS
TXT record, per cloud, pointing to the corresponding IDP domain. Specifying the
cloud identifier in the record supports ADNs configuring different IDPs for
different clouds. Note that since accounts are expected to change their IDP
rarely, clients can cache this mapping aggressively.

For example, if an account hosted on the Zoom commercial cloud is using 
\texttt{example.org} as their ADN and \texttt{generic-idp.com} as their IDP,
then the DNS entry for \texttt{example.org} should include a TXT record
with a value like:
\begin{center}
    \texttt{v=zoomadn us.zoom.idp.commercial=examplecorp.generic-idp.com}
\end{center}


As part of the process of validating IDP attestations (see
Section~\ref{subsec:validateattestation}), clients request the IDP domain value
from the Zoom server, and compare it with the value returned in the account's
DNS TXT record. In addition, once the ZTT is deployed, clients will also check
that this information matches what is in the ZTT for auditing purposes. If the
values do not match, then clients won't complete verification of or display any
identifiers for the account's users.

\subsection{IDP Attestations}

IDP attestations are generated and verified according to the OpenID Connect (OIDC) protocol.
OIDC is an extension of the widely used OAuth 2.0 authentication protocol, an industry standard that
many IDPs and Single Sign-On providers already support. OIDC provides a standardized format, the
ID token, to express claims about identities and their attributes. It also specifies how users can
request attestations for their own identity and verify ones obtained from other users.

We customize the protocol by:

\begin{enumerate}
\item Introducing an additional attribute \texttt{"zoom-identity-snapshot"} to the ID token in order
    to encode the state of a user's identity on Zoom. The IDP keeps track of the latest value of
    this attribute for every account user.
\item Specifying how this attribute can only be updated by the authorized user, and not by any other
    user or entity, including Zoom servers.
\item Specifying how this customized ID token for a specific user identity can be validated by other users.
\end{enumerate}

Our modified OIDC ID token (which we will also refer to as an \textit{IDP
attestation}) is a signed JSON Web Token (JWT) data structure which contains
claims about a user’s Zoom identity. The payload might
look like the following:

\begin{Verbatim}
{
  "iss": "https://examplecorp.generic-idp.com", // issuer

  "email": "alice@example.com",

  "zoom-identity-snapshot": "409788…",

  "exp": 1311281970 // expiration time
  "iat": 1311280970, // issue time

  [...]
}
\end{Verbatim}

The token contains an \texttt{issuer} field which identifies the OIDC issuer of the token (the IDP). In
order for Zoom clients to accept an identity claim, the \texttt{issuer} field must match
the IDP domain associated with the account.
\texttt{iat} and \texttt{exp} specify the validity of the token. The
\texttt{zoom-identity-snapshot} field encodes the state of a user’s identity,
such as the user's sigchain-backed identity; it is treated by the IDP as
an opaque string, and the IDP does not need to check its validity (the identity
snapshot is further described in Section~\ref{subsec:snapshots}).  

The fetched attestation may be shared with and validated by anyone, so it
doesn't include the \texttt{aud} field. Unlike standard OIDC ID tokens, which
are ``bearer tokens'' in that they can be presented by a user to prove their
identity to the party indicated in the token's \texttt{aud} field, our
attestations are meant to be exchanged with potentially many other clients to
prove a binding of a particular client's identity to their keys. Because
compliant OIDC implementations enforce that the \texttt{aud} field contains
their own identifier, they will not accept our attestations (which lack this
field) as a standard ID token for authentication purposes.

The \texttt{email} value must match the email address that the user logged into
Zoom with (and that the server gives to the verifying client) in order for validation to succeed.  Currently, the
\texttt{email} field is required; in the future, we might make this field
optional to allow for proving that a user is a member of a specific account
without revealing their full identity.

\subsection{Updating Snapshots}\label{subsec:generateattestation}
In order for a user to receive a valid OAuth access token to read and update their
\texttt{zoom-identity-snapshot} attribute, they must successfully complete the
OAuth 2.0 Authorization Code Flow with PKCE~\cite{rfc7636} with their IDP for the same email address they
use to login to Zoom. This is implemented as a second, separate flow from the
Zoom user login flow. To ensure that only an authorized user on a Zoom native
client is able to update the identity snapshot stored by the IDP, our protocol
requires IDPs to:

\begin{enumerate}
\item Introduce new OAuth scopes \texttt{idpSnapshot.manage} (required
    to update one's snapshot) and \texttt{idpSnapshot.read} (required to
    read one's snapshot).
\item Issue access tokens with the \texttt{idpSnapshot.manage} scope only for requests that use
    PKCE, and where the redirect URI is one of the fixed custom URIs intended to refer to the native Zoom desktop and mobile clients (such as \texttt{zoommtg://zoom.us/oauth2}).
\item Issue access tokens with the \texttt{idpSnapshot.read} scope, regardless of the specific OIDC flow. This scope supports future use cases where the Zoom server can fetch valid user attestations without modifying the attested keys.
\end{enumerate}

With the custom URI redirect, we trust the operating system and browser to redirect to the native
Zoom app and not to a website in a browser: such a website might be serving malicious JavaScript
from a compromised web server that could hijack the authorization flow. PKCE is an OAuth 2.0
extension that prevents other apps installed on the user's device from intercepting the
authorization code. The Zoom app will not share the resulting ``write" access token with anyone
else, including the Zoom server, but read-only access to snapshots can be extended to all access
tokens, including those issued to browser sessions.

After the client has obtained an access token which includes the \texttt{idpSnapshot.manage} scope, they
can use it for a dedicated IDP API endpoint to request an attestation with an arbitrary
\texttt{zoom-identity-snapshot} value.

We realize that the protections given to write ID tokens depend on the security of the underlying
platform including the user's browser, their OS and their hardware, but we intend these protections
to be best effort measures.

\subsection{Validating IDP Attestations}\label{subsec:validateattestation}

IDP attestations can be validated like standard OIDC ID tokens in the Authorization Code
Flow~\cite{oidc} with a few modifications. Users must:

\begin{enumerate}
\item Verify that the IDP domain in the \texttt{iss} field of the JWT matches
    the IDP in the ADN's DNS TXT record and the IDP returned by the Zoom server.
    This ensures that the IDP is authorized to sign on
    behalf of the account ADN (as specified in Section~\ref{subsec:adnToIdp}).
\item Use OpenID Connect Discovery to ensure that
the key used to sign the JWT is valid. To do this, make a request to \\
    \texttt{https://examplecorp.generic-idp.com/.well-known/openid-configuration}) \\
    for the JWKS, the keys used to sign the OIDC ID token.
\item Validate the JWT, including checking its signature and expiration date.
\item Validate that the \texttt{zoom-identity-snapshot} value and email address match what is provided by the Zoom server.
\end{enumerate}

We take several measures to improve the security and privacy of the attestation
validation process.

To ensure confidentiality and authenticity of DNS queries in step 1, clients
obtain the DNS TXT record by making a DNS over HTTPS (DoH) request to the DNS
resolver 1.1.1.1 (operated by Cloudflare), using Cloudflare's JSON-formatted DoH
API endpoint.

We have deployed Zoom-hosted proxy servers, described below, to hide client IPs
from external servers in steps 1 and 2. However, a client's proxy and network
configurations may prevent the client from successfully connecting to our
Zoom-hosted proxy. If the Zoom client detects a proxy, it will attempt sending
requests both via the detected proxy (which reveals its proxy's IP to the target
domain), and via the Zoom proxy. To reduce latency, the client will send the
requests simultaneously and use the response that returns first. If neither
request succeeds, then attestation validation will fail. 

In more detail, the client uses the Zoom-hosted proxy servers in step 1 when
sending a DoH request to a supported DNS resolver, and in step 2 when requesting
JWKS belonging to other users' IDPs. The client first establishes an HTTPS
connection to the proxy service, over which it sends a \texttt{HTTP CONNECT}
request to establish a tunneled connection to the target domain. The proxy
server authenticates the request, and also confirms that the target domain is an
expected domain (e.g. that it is a subdomain of a known IDP domain). The client
now establishes a HTTPS connection with the target server through the tunnel.
This allows the client to use TLS to fetch the requisite data, but not reveal
the client IP address to network eavesdroppers or the target domain. 
In order to mitigate abuse, the proxy checks that the target domain (in the TLS
ClientHello) and the target port (in the HTTP CONNECT) are expected, but is not
otherwise involved in the standard TLS handshake and certificate validation
process for the target domain.


\subsection{Zoom Identity Snapshots}
\label{subsec:snapshots}

The \texttt{zoom-identity-snapshot} field of an IDP attestation binds the email and ADN to the
set of cryptographic keys that the client will use in their interactions and, in some cases, to
their whole cryptographic identity by including the user and account sigchain tails.

For example, in the context of E2EE meetings
(Section~\ref{subsec:idpattestationmeetings}), the snapshots currently include
only the identity verification key $\ivk$ used by the device posting the
attestation to join that specific meeting. In the future, once we start
leveraging sigchains for identity in E2EE meetings, we will also include the
sigchain tails in the snapshot. 

There are several advantages of binding to the full cryptographic user identity:
devices will be able to share a single attestation, which would need to be
updated only when it expires (a configurable interval) or when there are changes
to the user's identity, thus simplifying interactions with identity providers
and improving the user experience. Note that these attestations of sigchain
tails are not considered valid if the sigchain's user has not reviewed all
changes to their identity (e.g. reviewed newly added devices). In addition,
because these attestations would cover the full history of a user's
cryptographic identity, any past misbehavior by the
server is more likely to be detected.

\subsection{Security Properties}\label{subsec:idpsecurity}

When clients verify an IDP attestation, they obtain the following informal
guarantee: the user who claims to control the cryptographic keys included in the
attestation has successfully authenticated to the given IDP (w.r.t. to the email
address in the attestation), and the owner of the attestation's ADN domain has
delegated the IDP to make these statements on the account's behalf.

This property relies on several assumptions. Beyond the security of the
cryptographic primitives, the verifier relies on the web PKI to obtain the ADN's
DNS TXT entry from the DNS resolver Cloudflare over DoH, as well as the IDP's
public keys used to verify the attestations.
A fraudulent attestation may also be produced upon compromise of the attested user's
device or of the user's authentication credentials (including any two-factor
authentication, if enforced by the IDP), or by compromising the IDP's server infrastructure.

Clients verify Zoom's suggestion for an ADN's preferred IDP through a DNS 
query. While relying on TLS (similarly to how JWKs are fetched) or DNSSEC 
would prevent some attacks from compromised DNS providers or network 
attackers, using a DNS TXT record eases the setup for our customers. We
rely on Cloudflare as a trusted DNS resolver to confirm that the IDP indicated
by Zoom is indeed the one the ADN owner intends to use. The DNS resolver can
convince clients that a certain account is not using an IDP by e.g., returning a
DNS NXDOMAIN message, or a wrong IDP value: in this case, because an honest Zoom
and the ADN have returned different IDP domain values, the client will not
consider the attestation valid and will not display the related information in
the user interface. We stress that control of the Zoom server infrastructure
alone is not sufficient to convince clients of a wrong IDP value. In the future, with the
transparency layer offered by the ZTT (Section~\ref{sec:ztt}), any server
tampering of an account’s ADN or IDP will be detected. We plan to offer users
other options for DNS resolvers in addition to 1.1.1.1 in the future.

From a privacy perspective, users who opt into sharing attestations share a
binding of their email address and the long-term device key(s) used by their
devices, and thus reveal their user, account and device identifiers to their
communication partners. 

We attempt to minimize the information disclosed to third-party servers: by
using the Zoom-hosted HTTPS CONNECT proxies, the verifying client can avoid
leaking its client IP to supported DoH resolvers and other users' IDPs (for
example, if a user who does not have an IDP is verifying another user's
attestation from Okta, their requests through the proxy would hide the client IP
from Cloudflare and Okta). As previously described, if the client has a proxy configured, 
its proxy's IP address may be revealed to the target domain.

The concrete properties users obtain when leveraging IDP attestations depend on the specific
application. For example, in an E2EE meeting (Section~\ref{subsec:idpattestationmeetings}), users
are able to confirm that their meeting partners belong to a specific account (identified by a domain
they know) even if the users have never previously met. Meeting hosts can leverage attestations to
prevent MitM attacks without manually checking security codes. We discuss how attestations are
leveraged in E2EE Zoom Meetings in Section~\ref{subsec:idpattestationmeetings}, and plan to extend
support to other products in the future.