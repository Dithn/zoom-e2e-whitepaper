\section{Understanding Multiple Devices}
\label{appendix:multidev}

As detailed in Section~\ref{p2:multidev}, device graphs can become complicated; we introduce a formal 
model to reason about them and what guarantees are available.

\begin{definition}
A \textbf{device family state} is a tuple: 
\begin{gather*}
\zfam_t = \langle \zset{d}_t, \zdevtime_t, \zset{r}_t, \zset{a}_t, \zset{p}_t, \zset{b}_t \rangle
\end{gather*}
Where:
\begin{itemize}
\item $t \in \integers_0$ is a non-negative integer which is used to order states in a sequence, which we refer to as ``time''.
\item $\zset{d}_t$ is a finite set whose elements represent the devices at time $t$.
\item $\zdevtime_t : \zset{d}_t \rightarrow \integers_0$ is a function that maps devices to
the time $t$ that they were provisioned.
\item $\zset{r}_t \subseteq \zset{d}_t$ is the set of revoked devices at time $t$. $\zdev{x} \in \zset{r}_t$ means that $\zdev{x}$ is revoked at time $t$ or before.
\item $\zset{a}_t \subseteq \zset{d}_t \times \zset{d}_t$ is the set of device approvals at
time $t$. $\langle \zdev{x}, \zdev{y} \rangle \in \zset{a}_t$ means that $\zdev{x}$ approved
$\zdev{y}$ at time $t$ or before.
\item $\zset{p}_t$ is the set of per-user-keys (PUKs).
\item $\zset{b}_t \subseteq \zset{p}_t \times \zset{d}_t \times \zset{d}_t$ is the set of PUK
boxes, where one device boxes the private keys of a PUK for itself or another device.
That is, $\langle \zkey{k}, \zdev{x}, \zdev{y} \rangle \in \zset{b}_t$ means that 
device $\zdev{x}$ boxed $\zkey{k}$ for device $\zdev{y}$ at time $t$ or before.
\end{itemize}
\end{definition}

It’s also worth defining, for convenience, a function that maps a device $\zdev{x}$
to all the PUKs that it knows. Recall that $\zdev{x}$ knows a PUK $\zkey{k}$
if another device $\zdev{y}$ boxed $\zkey{k}$ for it; or symbolically,
that $\langle \zkey{k}, \zdev{x}, \zdev{y} \rangle \in \zset{b}_t$.
Note that in some cases, $\zdev{x} = \zdev{y}$. 

\begin{definition}
Let $\zpuks_t : \zset{d}_t \rightarrow \zset{p}_t$ be defined as:
\[
  \zpuks_t(\zdev{x}) = \left\{
  \zkey{k} \, | \,
\exists \zdev{y} \in \zset{d}_t \text{ such that }
 \langle \zkey{k}, \zdev{y}, \zdev{x} \rangle \in \zset{b}_t
\right\}
\]
\end{definition}

We now define how a device family transitions from one configuration to
another over time.

\begin{definition}
  \label{d:trans}
  A \textbf{device family} consists of a sequence $\zfam = (\zfam_0,\zfam_1,\dots,\zfam_n)$. Each of the $\zfam_i$ is a family state as defined above. $\zfam_0$ is the empty family state (all the components of the tuple are empty sets or functions with empty domain). Moreover, we require that the difference between any two consecutive states can be seen as the result of one of the following transitions below (which leave the components of the tuple which are not explicitly updated as unchanged):

\begin{enumerate}
\item $\deviceadd{}$. Device $\zdev{x}$ is provisioned at time $t$:
 \begin{itemize}
   \item[-] Preconditions:
    \begin{enumerate}
     \item $\zdev{x} \notin \zset{d}_{t-1}$
    \end{enumerate}
   \item[-] Effects:
    \begin{enumerate}
      \item $\zset{d}_t \leftarrow \zset{d}_{t-1} \cup \left\{ \zdev{x} \right\}$
      \item $\zdevtime_t \leftarrow \zdevtime_{t-1} \cup \left\{ \langle \zdev{x}, t \rangle \right\}$
      \item $\zset{P}_t \leftarrow \zset{p}_{t-1} \cup \left\{ \zkey{k} \right\}$ for some new key $\zkey{k}$
        \label{e:prov:newpuk}
        \item $\zset{b}_t \leftarrow \zset{b}_{t-1} \cup \left\{ \langle \zkey{k}, \zdev{x}, \zdev{y} \rangle \;|\; \zdev{y} \in \zset{d}_t \setsub \zset{r}_t \right\}$
          \label{e:prov:box}
    \end{enumerate}
 \end{itemize}
\item $\devicerevoke{}$. Device $\zdev{x}$ is revoking devices $\zset{S}$ at 
    time $t$:
    \begin{itemize}
      \item[-] Preconditions:
      \begin{enumerate}
        \item $\zdev{x} \in \zset{d}_{t-1}$
        \item $\zdev{x} \notin \zset{r}_{t-1}$
        \item $\zset{s} \subseteq \zset{d}_{t-1}$
        \item $\zset{s} \cap \zset{r}_{t-1} = \varnothing$
      \end{enumerate}
      \item[-] Effects:
      \begin{enumerate}
        \item $\zset{r}_t \leftarrow \zset{r}_{t-1} \cup \zset{s}$
        \item $\zset{p}_{t} \leftarrow \zset{p}_{t-1} \cup \left\{ \zkey{k} \right\}$
                for some new PUK $\zkey{k}$
         \item $\zset{b}_t \leftarrow \zset{b}_{t-1} \cup \left\{ \langle \zkey{k}, \zdev{x}, \zdev{y} \rangle \;|\; \zdev{y} \in \zset{d}_t \setsub \zset{r}_t \right\}$
              \label{e:revoke:box}
      \end{enumerate}
    \end{itemize}
    Note that our model also allows the server to initiate a $\devicerevoke{}$, in which case the PUK will not be rotated. We don't model this transition here for simplicity.
\item $\peruserkeyrotate{}$. Device $\zdev{x}$ rotates PerUserKey at
  time $t$ (usually after another device self-revoked):
  \begin{itemize}
    \item[-] Preconditions:
    \begin{enumerate}
      \item $\zdev{x} \in \zset{d}_{t-1}$
      \item $\zdev{x} \notin \zset{r}_{t-1}$
   \end{enumerate}
    \item[-] Effects:
    \begin{enumerate}
      \item $\zset{p}_t \leftarrow \zset{p}_{t-1} \cup \left\{ \zkey{k} \right\}$ 
            for some new PUK $k$.
        \item $\zset{b}_t \leftarrow \zset{b}_{t-1} \cup \left\{ \langle \zkey{k}, \zdev{x}, \zdev{y} \rangle \;|\; \zdev{y} \in \zset{d}_t \setsub \zset{r}_t \right\}$
    \end{enumerate}
  \end{itemize}
\item $\batchapprove{}$. Device $\zdev{x}$ approves devices at time
   $t$, meaning it approves all non-revoked devices $\zdev{y}$ such that 
   $\zdevtime_t(\zdev{x}) < \zdevtime_t(\zdev{y})$.
   \begin{itemize}
     \item[-]Preconditions:
     \begin{enumerate}
       \item $\zdev{x} \in \zset{d}_{t-1}$
       \item $\zdev{x} \notin \zset{r}_{t-1}$
     \end{enumerate} 
     \item[-]Effects: 
     \begin{enumerate}
      \item $\zset{a}_t \leftarrow \zset{a}_{t-1} \cup  
      \left\{ \langle \zdev{x}, \zdev{y} \rangle  \;|\; \zdev{y} \in \zset{d}_{t} \setsub \zset{r}_{t},
      \zdevtime_t(\zdev{x}) < \zdevtime_t(\zdev{y})\right \}$
       \label{e:approve:add_to_a}
       \label{e:approve:devices}
      \item $\zset{b}_{t} \leftarrow \zset{b}_{t-1} \cup
               \left\{ \langle \zkey{k}, \zdev{x}, \zdev{y} \rangle 
               \;|\; \zdev{y} \in \zset{d}_{t} \setsub \zset{r}_{t},
               \zdevtime_t(\zdev{x}) < \zdevtime_t(\zdev{y}),
               \zkey{k} \in \zpuks_{t-1}(\zdev{x})
               \right\}$
               \label{e:approve:box}
     \end{enumerate}
   \end{itemize}
\end{enumerate}

\end{definition}

The thinking behind the batch approval operation is that a user should always
revoke all devices when necessary, and therefore should be approving all the
devices that can be approved whenever they approve any device.

$\devicekeyrotate{}$ is not explicitly modeled here, as it would have the same preconditions and effects of a $\peruserkeyrotate{}$ link.

Note that $\zset{a}_t$ defines an undirected graph on the set $\zset{D}_t$ (where elements of $\zset{a}_t$ are the edges). 

\begin{definition}
  We denote with $\zeq_t(\zdev{x}) \subset \zset{D}_t$ the connected component of $\zdev{x}$ in the graph defined on $\zset{D}_t$ by $\zset{A}_t$. 
  Let $\zverif_t : \zset{d}_t \rightarrow \integers$ be the function defined as:
\[
\zverif_{t}(\zdev{x}) = \min \left\{
      \zdevtime_t(\zdev{y})  
 \;|\; \zdev{y}  \in \zeq_t(\zdev{x}) 
     \right\}
\]
\end{definition}

We can define an equivalence relation between two devices with respect to a device family state $\zfam_{t}$ (which boils down to being part of the same connected component):

\begin{definition}
  Given a device family $\zfam_{t}$, $\forall \zdev{x},\zdev{y} \in \zset{D}_t$ 
  define $\zdev{x} \zequiv{t} \zdev{y}$ 
  iff $\zverif_t(\zdev{x}) = \zverif_t(\zdev{y})$.
\end{definition}

Note that, while revoked devices cannot perform new approvals, the approvals they make before being revoked are still considered part of the graph. 
For example, consider a family with 3 devices and the following transitions: 
\begin{enumerate}
  \item $\zdev{a}$ is provisioned
  \item $\zdev{b}$ is provisioned
  \item $\zdev{a}$ approves $\zdev{b}$
  \item $\zdev{c}$ is provisioned
  \item $\zdev{b}$ approves $\zdev{c}$
\end{enumerate}
At this point ($t=5$), we have:
$\zverif_5(\zdev{a}) = \zverif_5(\zdev{b}) = \zverif_5(\zdev{c}) = 1$.
Then (at $t=6$) $\zdev{b}$ self-revokes. $\zverif_6(\zdev{c})$ remains at $1$,
even though the only path from $\zdev{a}$ to $\zdev{c}$ at $t=6$
crosses $\zdev{b}$, a revoked device.

It’s worth noting that by definition of $\zverif_t$ and the equivalence
$\zequiv{t}$, the idea of approval is bidirectional. That is, if Alice
provisions $\zdev{x}$ then $\zdev{y}$, then approves $\zdev{y}$ with
$\zdev{x}$, then $\zdev{x}$ and $\zdev{y}$ are in the same equivalence class,
even though she never approved $\zdev{x}$ with $\zdev{y}$. We think this is a
useful simplification. The rationale is that when $\zdev{y}$ was added, Alice 
had access to the list of her devices on $\zdev{y}$ which includes $\zdev{x}$, 
so we assume she implicitly approved $\zdev{x}$ with $\zdev{y}$, since she 
didn't revoke it when she had a chance.

\subsection{A Claim About Device Equivalence Classes}

We have an important claim to flesh out: that all devices in the same
equivalence class know the same PerUserKey secrets. This allows us to
treat them the same throughout the UI, either for the purposes of
propagating trust, or for the purposes of propagating secrets.

\begin{theorem}
 For any device family, and any time $t$, $\forall \zdev{x}, \zdev{y} \in \zactive{t}$,
  if $\zdev{x} \zequiv{t} \zdev{y}$ then $\zpuks_t(\zdev{x}) = \zpuks_t(\zdev{y})$.
\label{thm:know_puks}
\end{theorem}

Before we get to the proof, it helps to proves some simpler lemmas about
the structure of device families given our transition rules in Definition~\ref{d:trans}.

First, a simple observation:

\begin{lemma} 
  For any device family, any time $t$, $\forall \zdev{x},\zdev{y} \in \zactive{t}$,
  if $\langle \zdev{x}, \zdev{y} \rangle \in \zset{a}_t$ then
  $\zverif_t(\zdev{x}) = \zverif_t(\zdev{y})$.
  \label{l:edge}
\end{lemma}

\textbf{Proof}. If if $\langle \zdev{x}, \zdev{y} \rangle \in \zset{a}_t$, then the two are
 connected, and thus $\zeq_t(\zdev{x}) = \zeq_t(\zdev{y})$, from which we have $\zverif_t(\zdev{x})
 = \zverif_t(\zdev{y})$. $\blacksquare$

The second lemma states that older devices know strictly more PerUserKey
private halves than newer devices. For devices, $\zdev{a}, \zdev{b}, \zdev{c}$
and $\zdev{d}$, provisioned in that order, it is clear to see that the later
devices will always box for the earlier devices. But keep in mind that 
when $\zdev{b}$ approves new devices, it never sends PerUserKeys back to
earlier devices. So we must formally show that, for instance, $\zdev{c}$
approving $\zdev{d}$ does not transmit PerUserKeys from $\zdev{a}$ 
that $\zdev{b}$ didn't know about.

\begin{lemma}
  For any device family, at any time $t$, $\forall \zdev{x},\zdev{y} \in \zset{d}_t
\setsub \zset{r}_t$, if $\zdevtime_t(\zdev{x}) <
\zdevtime_t(\zdev{y})$ then $\zpuks_t(\zdev{y}) \subseteq
\zpuks_t(\zdev{x})$.
 \label{l:puks}
\end{lemma}

\textbf{Proof}.
%
The proof is by induction over time $t$. For the base case, for any device family containing less
than two devices (which includes any family at time $t=0$), the claim is trivially true.
For the inductive case, assume the lemma is true for time $t-1$.
Take arbitrary $\zdev{x},\zdev{y} \in \zset{d}_t \setsub \zset{r}_t$, such that $\zdevtime_t(\zdev{x}) <
\zdevtime_t(\zdev{y})$.  Consider the 4 possible state transitions:
%
\begin{enumerate} 
  \item $\deviceadd{}$. A new device $\zdev{z}$ is introduced at time $t$, meaning
$\zdevtime_t(\zdev{z}) = t$. Note that it cannot be $\zdev{z} = \zdev{x}$ as $\zdevtime_t(\zdev{x})
< \zdevtime_t(\zdev{y}) \leq t$.

If $\zdev{z} = \zdev{y}$, then let $\zkey{k}$ be the PUK generated at Effect~\ref{e:prov:newpuk}.
$\zset{B}_t$ gets $\langle \zkey{k}, \zdev{y}, \zdev{y} \rangle$ and $\langle \zkey{k}, \zdev{y},
\zdev{x} \rangle$ as a result of Effect~\ref{e:prov:box}. Thus, $\zpuks_t(\zdev{y}) = \left\{
\zkey{k} \right\}$ and $\zkey{k} \in \zpuks_t(\zdev{x})$, which proves $\zpuks_t(\zdev{y}) \subseteq
\zpuks_t(\zdev{x})$.

If instead $\zdev{z} \not\in \{\zdev{x},\zdev{y}\}$, it must be that $\zdevtime_t(\zdev{x}) <
\zdevtime_t(\zdev{y}) < \zdevtime_t(\zdev{z})$. By inductive assumption, $\zpuks_{t-1}(\zdev{y})
\subseteq \zpuks_{t-1}(\zdev{x})$. As before, let $\zkey{k}$ be the PUK that is introduced at
Effect~\ref{e:prov:newpuk}. At Effect~\ref{e:prov:box}, $\zset{b}_t$ is augmented with $\langle
\zkey{k}, \zdev{z}, \zdev{y} \rangle$ and $\langle \zkey{k}, \zdev{z}, \zdev{x} \rangle$, and boxes
for other devices. Thus, $\zpuks_t(\zdev{x}) = \zpuks_{t-1}(\zdev{x}) \cup \left\{ k \right\}$, and
$\zpuks_t(\zdev{y}) = \zpuks_{t-1}(\zdev{y}) \cup \left\{ k \right\}$. Combining with the inductive
assumption, it follows that $\zpuks_t(\zdev{y}) \subseteq \zpuks_t(\zdev{x})$.

 \item $\devicerevoke{}$ and $\peruserkeyrotate{}$ follow the
 same logic as device provisioning, so we leave out the argument for brevity.

 \item $\batchapprove{}$ is the interesting case. Device
  $\zdev{z}$ approves all younger devices at time $t$. We consider three
  disjoint subcases:
  \begin{enumerate}
    \item $\zdevtime_t(\zdev{z}) \leq \zdevtime_t(\zdev{x}) < \zdevtime_t(\zdev{y})$.
      As a result of Effect~\ref{e:approve:box}, we get that
        $\zpuks_t(\zdev{x}) = \zpuks_{t-1}(\zdev{x}) \cup \zpuks_t(\zdev{z})$ and
        $\zpuks_t(\zdev{y}) = \zpuks_{t-1}(\zdev{y}) \cup \zpuks_t(\zdev{z})$.
        By inductive assumption we have that 
        $\zpuks_{t-1}(\zdev{y}) \subseteq \zpuks_{t-1}(\zdev{x})$. Combining
        these three statements proves 
         $\zpuks_t(\zdev{y}) \subseteq \zpuks_t(\zdev{x})$.

    \item $\zdevtime_t(\zdev{x}) < \zdevtime_t(\zdev{z}) < \zdevtime_t(\zdev{y})$.
      Here we apply the inductive assuption to $\zdev{x}$ and $\zdev{z}$, giving
      us that $\zpuks_{t-1}(\zdev{z}) \subseteq \zpuks_{t-1}(\zdev{x})$.
      Effect~\ref{e:approve:box} doesn't change $\zpuks_{t-1}(\zdev{x})$
      so therefore $\zpuks_{t-1}(\zdev{x}) = \zpuks_{t}(\zdev{x})$.
      Chaining these set relations together gives 
      us that $\zpuks_{t-1}(\zdev{z}) \subseteq \zpuks_{t}(\zdev{x})$.
      Next, apply the inductive assumption to $\zdev{z}$ and $\zdev{y}$, giving
      us that $\zpuks_{t-1}(\zdev{y}) \subseteq \zpuks_{t-1}(\zdev{z})$. 
      By Effect~\ref{e:approve:box}, $\zpuks_{t}(\zdev{y}) = \zpuks_{t-1}(\zdev{y})
      \cup \zpuks_{t-1}(\zdev{z})$. By basic set theory, it is still the case
      that $\zpuks_{t}(\zdev{y}) \subseteq \zpuks_{t-1}(\zdev{z})$. By 
      transitivity, $\zpuks_{t}(\zdev{y}) \subseteq
      \zpuks_{t}(\zdev{x})$.

    \item $\zdevtime_t(\zdev{x}) < \zdevtime_t(\zdev{y}) \leq \zdevtime_t(\zdev{z})$.
      Due to Effect~\ref{e:approve:devices}, devices $\zdev{y}$ and
      $\zdev{x}$ do not receive any new keys, i.e.
      $\zpuks_{t}(\zdev{x}) = \zpuks_{t-1}(\zdev{x})$ and
      $\zpuks_{t}(\zdev{y}) = \zpuks_{t-1}(\zdev{y})$. Combining
      with the inductive assumption
       $\zpuks_{t-1}(\zdev{y}) \subseteq \zpuks_{t-1}(\zdev{x})$ proves
       the subcase. $\blacksquare$
  \end{enumerate}
\end{enumerate}

The next lemma establishes that older devices
will always be grouped into older equivalence classes:

\begin{lemma}
  For any device family $\zfam$, any time $t$, $\forall \zdev{x},\zdev{y} \in \zset{d}_t \setminus \zset{R}_t$, 
  if $\zdevtime_t(\zdev{x}) < \zdevtime_t(\zdev{y})$ then 
  $\zverif_t(\zdev{x}) \le \zverif_t(\zdev{y})$.
  \label{l:verif_ineq}
\end{lemma}

\textbf{Proof}. Let $\zdev{k} \in \zeq_t(\zdev{y})$ be such that
$\zdevtimet{t}{k}$ is minimal (in particular, $\zverift{t}{y} = \zdevtimet{t}{k}$). We have 3 cases.

Consider the case $\zdevtimet{t}{k} < \zdevtimet{t}{x}$ (which implies $\zdev{k} \neq \zdev{y}$, as
$\zdevtimet{t}{x} < \zdevtimet{t}{y}$). Consider a path from $\zdev{k}$ to $\zdev{y}$ along edges in
$\zset{a}_t$. There must be some $\langle \zdev{a}, \zdev{b} \rangle \in \zset{a}_t$ along this path
such that $\zdevtimet{t}{a} < \zdevtimet{t}{x} \le \zdevtimet{t}{b}$. This implies that $\langle
\zdev{a}, \zdev{x} \rangle \in \zset{a}_t$ (as $\zdev{a}$ must have approved $\zdev{x}$ when it
approved $\zdev{b}$), which means $\zdev{x} \in \zeq_t(\zdev{y})$, and therefore $\zverift{t}{y} =
\zverift{t}{x}$.

If $\zdevtimet{t}{k} = \zdevtimet{t}{x}$, then $\zdev{x} = \zdev{k} \in \zeq_t(\zdev{y})$ and
$\zverift{t}{y} = \zverift{t}{x}$.

Finally, if $\zdevtimet{t}{x} < \zdevtimet{t}{k}$, we have that by definition $\zverift{t}{x} \leq \zdevtimet{t}{x}$ and thus 
$\zverift{t}{x} \leq \zdevtimet{t}{x} < \zdevtimet{t}{k} = \zverift{t}{y}$. $\blacksquare$

In the next lemma we show that a device performing an approval transition doesn't change its own
  equivalence class.

  \begin{lemma}
    For any device family, any time $t$, $\forall \zdev{x} \in \zactive{t}$, 
    if device $\zdev{x}$ performs approval at time $t$,
    then $\zverift{t}{x} = \zverift{t-1}{x}$.
    \label{l:nochange}
 \end{lemma}
 
 \textbf{Proof}.
 Assume for contradiction there exists a device family, and a time $t$ such that $\zdev{x}$ does a
 device approval transition at time $t$ and $\zverift{t}{x} < \zverift{t-1}{x}$. Pick $\zdev{i}$ to
 be minimal in $\zdev{x}$'s equivalence class at time $t$, i.e. such that $\zdevtimet{t}{i} =
 \zverift{t}{x}$, and consider a path (without repeated nodes) from $\zdev{i}$ to $\zdev{x}$. Since
 this path does not exist in $\zset{A}_{t-1}$ (else it would be $\zverift{t-1}{x} \leq
 \zdevtimet{t}{i}$ which leads to a contradiction), any such path must contain one of the edges
 $\langle \zdev{x}, \zdev{b} \rangle \in \zset{A}_t \setminus \zset{A}_{t-1}$ (with $\zdevtimet{t}{x}
 < \zdevtimet{t}{b}$) added due to Effect~\ref{e:approve:add_to_a} in the transition at time $t$.
 Since the path does not have repeated edges, the sub-path between $\zdev{b}$ and $\zdev{i}$ must be
 composed of edges in $\zset{A}_{t-1}$, from which we derive $\zverift{t-1}{b} = \zdevtimet{t-1}{i}$.
 Putting it altogether, at time $t-1$ we have that $\zdevtimet{t-1}{x}< \zdevtimet{t-1}{b}$ but
 $\zverift{t-1}{x} >  \zverift{t}{x} = \zdevtimet{t-1}{i} = \zverift{t-1}{b}$ which contradicts
 Lemma~\ref{l:verif_ineq}. $\blacksquare$

Additionally, a device performing an approval cannot change the equivalence classes of older devices.

\begin{lemma}
  For any device family, any time $t$, $\forall \zdev{x}, \zdev{y} \in \zactive{t}$ such that
  $\zdevtimet{t-1}{x} < \zdevtimet{t-1}{y}$, if $\zdev{y}$ runs approval
  at time $t$ then $\zverift{t}{x} = \zverift{t-1}{x}$.
  \label{l:olders}
\end{lemma}

\textbf{Proof}.
%
The proof is similar to the one of the previous lemma. Assume for contradiction there exists a
device family, and a time $t$ such that $\zdev{y}$ does a device approval transition at time $t$ and
$\zverift{t}{x} < \zverift{t-1}{x}$ for some device $\zdev{x}$ such that $\zdevtimet{t}{x} <
\zdevtimet{t}{y}$. Pick $\zdev{i}$ to be minimal in $\zdev{x}$'s equivalence class at time $t$, i.e.
such that $\zdevtimet{t}{i} = \zverift{t}{x}$, and consider a path (without repeated nodes) from
$\zdev{i}$ to $\zdev{x}$. Since this path does not exist in $\zset{A}_{t-1}$ (else it would be
$\zverift{t-1}{x} \leq \zdevtimet{t}{i}$ which leads to a contradiction), any such path must contain
one of the edges $\langle \zdev{y}, \zdev{b} \rangle \in \zset{A}_t \setminus \zset{A}_{t-1}$ (with
$\zdevtimet{t}{x} < \zdevtimet{t}{b}$) added due to Effect~\ref{e:approve:add_to_a} in the
transition at time $t$. Since the path does not have repeated edges, the sub-path between $\zdev{b}$
and $\zdev{i}$ must be composed of edges in $\zset{A}_{t-1}$, from which we derive $\zverift{t-1}{b}
= \zdevtimet{t-1}{i}$. Putting it altogether, at time $t-1$ we have that
$\zdevtimet{t-1}{x}<\zdevtimet{t-1}{y}<\zdevtimet{t-1}{b}$ but $\zverift{t-1}{x} > \zverift{t}{x} =
\zdevtimet{t-1}{i} = \zverift{t-1}{b}$ which contradicts Lemma~\ref{l:verif_ineq}. $\blacksquare$

Finally, we can prove Theorem~\ref{thm:know_puks}. The proof is by induction
over $t$. It is trivially true at time $t=0$, since there is only the null
equivalence class. Assume it is true for time $t-1$, and we prove true tor time
$t$. We proceed casewise, reasoning over the four transition types that could
explain the transition from $t-1$ to $t$.

And now to the casewise analysis:
\begin{enumerate}
  \item $\deviceadd{}$.
    \label{case:prov}
  A new device $\zdev{z}$ is introduced at time $t$. In this case, since $\zdev{x} \zequiv{t} \zdev{y}$, 
  $\zdev{z}$ cannot be neither $\zdev{x}$ nor $\zdev{y}$, and also $\zdev{x} \zequiv{t-1} \zdev{y}$.  
  We need to prove $\zpukst{t}{x} = \zpukst{t}{y}$,
  but without loss of generality, we can prove one inclusion,
  and argue the other follows by symmetry. Given $\zkey{k} \in \zpukst{t}{x}$,
  we need to prove $\zkey{k} \in \zpukst{t}{y}$.  We consider these subcases:
  \begin{enumerate}
    \item $\zkey{k} \in \zpukst{t-1}{x}$.
    Then by inductive assumption, $\zkey{k} \in  \zpukst{t-1}{y}$.
    Since the set $\zset{b}_t$ only grows over time,
    it follows that $\zkey{k} \in \zpukst{t}{y}$, which proves the case.
    \item $\zkey{k} \notin \zpukst{t-1}{x}$.
    That is, there does not exist a $\zdev{w}$
    such that $\langle \zkey{k}, \zdev{w}, \zdev{x} \rangle \in \zset{b}_{t-1}$.
    Then it follows that at Effect~\ref{e:prov:box} above, such a member was
    introduced. Recall that $\zdev{x}, \zdev{y} \in \zactive{t}$
    because the equivalence relation is only defined over devices in $\zactive{t}$.
    The loop in Effect~\ref{e:prov:box} is over all devices in $\zactive{t}$,
    so $\zdev{x}$ and $\zdev{y}$ were both included.
    Thus, it follows that $\langle \zkey{k}, \zdev{w}, \zdev{y} \rangle \in \zset{b}_t$,
    and therefore that $\zkey{k} \in \zpukst{t}{y}$.
  \end{enumerate}

 \item $\devicerevoke{}$. This case follows much like Case~\ref{case:prov}
 (Device Provisioning), just above. In considering the two subcases
 $\zkey{k} \in \zpukst{t-1}{x}$ is argued the same way.
 In the second subcase, 
 $\zkey{k} \notin \zpukst{t-1}{x}$.
 Then the $\langle \zkey{k}, \zdev{w}, \zdev{y} \rangle$
 triple must have been introduced at Effect~\ref{e:revoke:box}, and by similar argument as above 
 $\langle \zkey{k}, \zdev{w}, \zdev{y} \rangle \in \zset{b}_t$,
 and therefore $\zkey{k} \in \zpukst{t}{y}$.

 \item $\peruserkeyrotate{}$.
  This case follows from the same reasoning as the other two cases above (provisioning and revocation).

 \item $\batchapprove${}.
 This is the most interesting case. Let $\zdev{w}$ be the device that’s doing
 the approval at time $t$. Without loss of generality, assume that 
 $\zdevtimet{t}{x} < \zdevtimet{t}{y}$. The three cases to consider is how 
 $\zdevtimet{t}{w}$ is interwoven with $\zdevtimet{t}{x}$ and $\zdevtimet{t}{y}$.
 \begin{enumerate}
   \item $\zdevtimet{t}{w} \le \zdevtimet{t}{x} < \zdevtimet{t}{y}$. By Lemma~\ref{l:puks},
    $\zpukst{t-1}{x} \subseteq \zpukst{t-1}{w}$ and $\zpukst{t-1}{y} \subseteq \zpukst{t-1}{w}$.
    As a result of Effect~\ref{e:approve:box}, $\zdev{x}$ and $\zdev{y}$ will
    learn about all of the PerUserKeys that $\zdev{w}$ knows. Combining with the prior
    observation, $\zpukst{t}{x} = \zpukst{t}{y} = \zpukst{t}{w}$.
   \item $\zdevtimet{t}{x} < \zdevtimet{t}{w} < \zdevtimet{t}{y}$.
   This implies $\zverift{t}{x} \leq \zverift{t}{w} \leq \zverift{t}{y}$ by Lemma~\ref{l:verif_ineq},
   and since by assumption $\zdev{x} \zequiv{t} \zdev{y}$, it must be $\zverift{t}{x} = \zverift{t}{w}
   = \zverift{t}{y}$. By Lemmas~\ref{l:nochange} and \ref{l:olders},  
   $\zverift{t}{x} = \zverift{t-1}{x}$ and 
   $\zverift{t}{w} = \zverift{t-1}{w}$, and by transitivity, 
   $\zverift{t-1}{x} = \zverift{t-1}{w}$, and therefore 
   $\zdev{x} \zequiv{t-1} \zdev{w}$. We apply the inductive assumption to deduce 
   that $\zpukst{t-1}{x} = \zpukst{t-1}{w}$. 
   Again $\zpukst{t-1}{x}$ isn’t affected by $\zdev{w}$’s approving newer devices,
   so $\zpukst{t-1}{x} = \zpukst{t}{x}$ and therefore 
   $\zpukst{t}{x} = \zpukst{t-1}{w}$ by transitivity. 
   By Lemma~\ref{l:puks}, we know that $\zpukst{t-1}{y} \subseteq \zpukst{t-1}{w}$.
   By Effect~\ref{e:approve:box}, $\zdev{y}$ gets all of $\zdev{w}$’s keys, 
   and $\zpukst{t-1}{w}$ is unchanged, so this gives 
   $\zpukst{t}{y} = \zpukst{t}{w} = \zpukst{t-1}{w}$. Chaining set equality, 
   $\zpukst{t}{x} = \zpukst{t}{y}$, which proves the case.
   \item $\zdevtimet{t}{x} < \zdevtimet{t}{y} \le \zdevtimet{t}{w}$.
   By assumption $\zverift{t}{x} = \zverift{t}{y}$, and by Lemmas~\ref{l:nochange} and \ref{l:olders},
   $\zverift{t-1}{x} = \zverift{t}{x}$ and $\zverift{t-1}{y} = \zverift{t}{y}$.
   Chaining, $\zverift{t-1}{x} = \zverift{t-1}{y}$ which means
   $\zdev{x} \zequiv{t-1} \zdev{y}$. Applying the inductive assumption,
   $\zpukst{t-1}{x} = \zpukst{t-1}{y}$.
   Device $\zdev{w}$ running approval has no effect on $\zpukst{t-1}{x}$
   or $\zpukst{t-1}{y}$. That is, $\zpukst{t-1}{x} = \zpukst{t}{x}$ 
   and $\zpukst{t-1}{y} = \zpukst{t}{y}$. Chaining,
   $\zpukst{t}{x} = \zpukst{t}{y}$ which proves it. $\blacksquare$

 \end{enumerate}


\end{enumerate}
