\subsection{Compromise Prevention for Device Provisioning}\label{subsec:realtime} \textbf{Note: }
The features in this section are not currently available. We plan to release them in future updates.

With the ZTT (see Section~\ref{sec:ztt}), we have a mechanism for users to detect sigchain
corruptions after-the-fact. For instance, if an insider adds a new device onto Alice's sigchain to
impersonate her in a meeting while she is offline, Alice will detect the attack after she comes back
online, since the attacker had to commit the malicious change to Alice's sigchain to the ZTT.
Ideally, Alice could prevent unauthorized device additions from happening in the first place. Note
that users whose identities are vouched for by a trusted third-party $\idp$
(Section~\ref{sec:idpattestations}) already have good protections as long as their $\idp$ is not
compromised. We want to offer a strong alternative to all users that does not rely on external
parties.

Recall that users currently login to Zoom via SSO, OAuth, or simple username and password. Their
devices' keys are authenticated by multiple signatures: a self signature, an optional approval
signature by an older device (Section~\ref{subsec:multidev}), a signature by Zoom (as part of the
inclusion in the ZTT, once it is deployed), and an optional attestation from the $\idp$.

Further validations of the device keys are possible. For example, users can sign new devices with
existing devices at the time they are first added (as opposed to in a later approval), potentially
through a QR-code-based flow similar to the one Keybase has now. One device shows a QR code of a
random session key, and the other scans the QR code; this establishes an end-to-end secure tunnel
between the two devices that a compromised server cannot interfere with. Over this tunnel, devices
can exchange public keys and cross-sign each other.

Alternatively, organizations can set up policies that require an IT administrator to sign off on
device additions. The new user device can display the hash of its public key in QR code form. The IT
administrator can scan it, signing the new device with their administrator's signing key.

Devices without the required signatures might be prevented from signing on behalf of the user or
rotating the user's PUKs, thus denying them access to any content encrypted for the user. They might
also trigger extra UI warnings to other participants when joining meetings, or even be excluded from
meetings marked ``high security.''
