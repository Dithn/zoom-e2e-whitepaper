Zoom offers a comprehensive communications platform consisting of a variety of products, including
Zoom Team Chat, Zoom Phone, Zoom Mail Service, Zoom Whiteboard, and Zoom Meetings.

For users of these products, Zoom provides software for desktop and mobile operating systems and
embeds software in Zoom Room devices. In this document, when we refer to ``Zoom clients'' or simply
``clients,'' we include all these various forms of packaging. Crucially, these are systems to which
we can deploy cryptographic software. In contrast, some Zoom products can also be accessed through
other systems which are unable to support custom cryptographic protocols, or cannot offer the same
level of security when doing so. For example, users can join a meeting by dialing in from a landline
phone (with no cryptography support), or from their web browsers (which can perform cryptography,
but make it easy for the server to surreptitiously provide tampered cryptographic code).

Our goal is to provide the best security protections across all of these devices, mindful of the
constraints that each environment poses, and without compromising the easy and seamless experience
that our customers expect. When analyzing the security of our products, we consider a wide range of
potential adversaries, namely:

\begin{description}
	\item {\bf Outsiders:} Individuals who are not authorized to access specific information or data
	streams (such as everyone who is not mentioned as a sender or recipient in a given email thread,
	or everyone except the participants of a specific meeting), and do not have access to non-public
	information related to it (e.g., user passwords, meeting passwords, IDs, SSO systems). These
	attackers may monitor, intercept, and modify network traffic, but do not have privileged access
	to Zoom's infrastructure.
	\item {\bf Participants:} Zoom users who are authorized to participate in a specific
	communication or have been granted access to some information: for example, meeting participants
	who can access a meeting, because they know the meeting's ID and password or exercise other
	qualifying credentials.
	\item {\bf Insiders:} Those who develop and maintain Zoom's server infrastructure and its cloud
	providers, as well as attackers who have gained (even partial) access to this infrastructure.
\end{description}

Against these adversaries, colluding or working independently, we seek the following security goals:

\begin{description}
	\item {\bf Confidentiality:} Only authorized participants should have access to the contents of
	end-to-end encrypted communications. If the Zoom product supports removing users from encrypted
	communication channels, then those users should no longer have access to those communications
	after they are removed.
	\item {\bf Integrity:} Those who are not authorized participants should have no ability to
	corrupt those end-to-end encrypted communications.
	\item {\bf Abuse Prevention:} When authorized participants engage in abusive behavior, there is
	an effective mechanism to report them to Zoom's safety team, to help prevent further abuse.
\end{description}

Broadly speaking, end-to-end encrypted products aim to achieve confidentiality and integrity goals
even against insiders: we aim to detect, and if possible, prevent any violations of these
properties. While prevention is preferable, we stress that even detection can be a powerful
deterrent: it allows users to monitor their communications and quickly move to limit any damage,
while demonstrating our commitment to be transparent about breaches.

All Zoom products also have many server-driven security and access control mechanisms, including
transport-layer encryption, meeting passwords, and granular user and account level permissions.
While these are not meant to protect against insiders with privileged access to our infrastructure
(and are not the focus of this document), they are highly effective against outsiders.

\subsection{Limitations}
\label{subsec:overall-limit}

Our encryption protocols offer strong guarantees for our customers. However, as with any security
system, there are limitations to our approach. There are certain classes of attacks and threats that
we deem out of scope, including:

\begin{description}
	\item {\bf Metadata and traffic analysis:} Insiders and outsiders may learn information about
		the timing, duration, and bandwidth usage patterns of end-to-end encrypted communications,
		as well as the list of participants and IP addresses.
	\item {\bf Software flaws:} Zoom's client code or the third-party libraries it links against can
		have bugs, or worse, intentional backdoors. Zoom's binary build procedures might become
		compromised. In these cases, there are no good cryptographic guarantees we can make. Zoom
		relies on extensive analyses by independent third party auditors to reassure customers in
		this domain.
	\item {\bf Third party compromise:} We leverage third-party Single Sign-Ons (SSOs) and Identity
		Providers ($\idp$s) to independently vouch for the identity of Zoom's users. Doing so moves
		the trust away from Zoom and to identity providers that many of Zoom's enterprise users
		already trust for sensitive identity operations. Where we do rely on SSOs and $\idp$s, the
		additional security guarantees they offer might be lost due to attacks on their
		infrastructure.
	\item {\bf Denial of service:} The encryption architecture does not offer any guarantees about
		availability of the service. Insiders can always refuse service to a specific user,
		including when they try to perform security critical operations such as revoking a
		compromised key to prevent further data from being encrypted and thus exposed.
\end{description}

In addition, we note that many Zoom products have rich feature sets and work on multiple platforms.
Some of these features and use cases are incompatible with our security guarantees. For example,
Zoom Mail Service can be used to send and receive emails from other email services that do not
support end-to-end encryption (see Section~\ref{sec:mailext}). Or, dial-in phones can be used to
join Zoom Meetings, but do not support end-to-end encryption. E2E security of the type described in
this document is not possible in settings where Zoom products interface with these existing
standards. Moreover, users can access some Zoom products through their web browser, and without
installing Zoom's client. Supporting E2EE for web users poses certain challenges: secure, long-term
storage for cryptographic private keys might be unavailable; and worse, malicious web servers could
serve backdoored source code to web clients with little chance of detection. We intend to
participate in the web standards development process to facilitate the creation of browsers upon
which we could offer dependable E2E security.

We will further comment on the specific properties and limitations for each of the products and
features described in this whitepaper in their respective sections.
