\subsection{Account Escrow}
\label{subsec:keyescrow}

Some of the data that organizations store on the Zoom platform is encrypted with keys that are only
known to the devices of the users authorized to access it; the data is not available to Zoom
servers. For example, emails between active users of the Zoom Mail Service (Section~\ref{sec:email})
and some phone calls between Zoom Phone clients (Section~\ref{subsec:phone}) are in most cases
encrypted using the sender and recipient's keys only. However, administrators of these organizations
may also wish to have access to this data, for example, to protect against accidental data loss (if
the account member loses their devices) or to support legal retention and discovery. We refer to
solutions that give such organizations cryptographic access to their members' encrypted data as
\textit{escrow}, and in this section we present Zoom's solution to this problem.

When an account enables escrow, all the account members (\emph{escrowees}) receive an unskippable
notification informing them that escrow is being enabled. Upon acknowledgement, each member adds a
``virtual'' escrow device (Section\ref{subsubsec:escrowkeys}) to their own user sigchain. This
escrow device's public keys corresponds to secret keys known to a set of devices which are
controlled by \emph{Escrow Administrators (EAs)}, account members designated with escrow
permissions. This enables all account members to automatically share their PUKs (and by extension
their encrypted data) with these admins.

\subsubsection{Escrow Administrators and the EA Sigchain}
\label{subsubsec:escrowadmins}

To enable escrow for their account, an account administrator assigns EA permissions
(Section~\ref{subsubsec:escrowadminperm}) to one or more users to designate them as EAs. EAs will
then be able to provision one or more of their devices with fresh \emph{EA device keys}, which
ultimately will be used to decrypt the encrypted content escrowed from all the account's users.

EA device keys are independent of the device keys listed on the EA's own user sigchain (and used to
encrypt their own data); instead, the EA enabled devices and their public keys are added to a
special user sigchain, which we call the \emph{EA sigchain}. EA sigchains follow all the rules of
regular user sigchains (Section~\ref{subsubsec:usersigchains}), but they have no associated email,
and devices listed in this chain can belong to different EAs (instead of a single user).

In addition, the server adds an \textsf{EscrowAdmin} link to the account sigchain referring to the
EA sigchain's $\userID$, so that account members can be on the same page on whether escrow is
enabled or not. Each EA sigchain is associated with at most one account, and vice versa.

When provisioning an EA's device to have access to all the account's data, the device generates
fresh EA device keys and adds a new link to the EA sigchain. These users' devices therefore have two
sets of device keys: one on the user's own sigchain, and one on the EA sigchain. Once it is approved
by an existing device on the EA sigchain, the device will receive all the PUKs associated with this
EA sigchain, allowing it access to all the escrowed data. Provisioned EA devices will display to
their corresponding EA users the list of active EA devices (which EAs can use to monitor which
devices have access to the account's data) and the EA sigchain fingerprint. Monitoring EA devices
and comparing fingerprints prevents MitM attacks. The fingerprint can be compared before approving a
new EA device, and shared with all the account members so that they can confirm it before enabling
escrow. After enabling escrow, account members can continue to monitor the EA fingerprint to ensure
that their escrow devices are rotated for every EA PUK rotation.

Given that the Zoom server is responsible for enforcing which users are EAs, and therefore can add
devices to the EA sigchain, the first device will enable lockdown mode on the EA sigchain (see
Section~\ref{subsubsec:lockdownmode}), and members' clients will enforce that this is the case. This
prevents an attacker compromising the Zoom server from generating new PUKs on the EA sigchain and
therefore gaining access to account members' data. Lockdown mode also avoids having to rotate the
PUK whenever a new device is added to the EA sigchain, which has a significant performance impact as
it would trigger a PUK rotation in all the account member's chains as well
(Section~\ref{subsubsec:puks}).

When an account administrator disables escrow, the Zoom server revokes all escrow admin management
permissions, adds a new link to the account sigchain with an empty escrow admin {\userID}, and
revokes all device keys on the EA's chain. As soon as clients are aware of this, either through a
server notification or from their own monitoring, they revoke the virtual escrow devices from their
own chains (and rotate their PUKs). EAs' devices will also revoke the EA device's secret keys from
durable storage. If the user is moved between different accounts, and the new account has escrow
enabled, the user will be notified of the new escrow being setup and given the opportunity to check
the new EA sigchain's fingerprint. If the former account has escrow enabled, those EAs will only
have access to the user's ciphertexts before the account change, while the user's devices will share
all known PUKs with the new EAs. As detailed in the Security Limitations section, we rely on the
server to enforce proper access control for cyphertexts when the encryption keys cannot be promptly
rotated due to the user's devices being offline, which can happen in both the above cases.

It is important that the EAs maintain access to at least one device in the oldest non-empty approval
class at all times, as otherwise escrowed data could be irreparably lost. For this reason,
administrators are required to generate a backup key on the EA sigchain when enabling escrow. If all
the devices on the EA's oldest class are unavailable, the only option to recover is to turn escrow
off and back on, which would trigger notifications to all account members to acknowledge escrow
again.

When escrow is enabled, or re-enabled when old escrow devices are unavailable, the new EAs will have
access to a specific user's data only after that user comes online to encrypt their keys for the
EAs. Therefore, there may be unrecoverable gaps in the content that can be recovered for the
escrowed users. To avoid this, we recommend account administrators to enable escrow before
encryption features that depend on the key management infrastructure we describe (such as Zoom Mail
Service and Advanced Encryption for Voicemail), to avoid turning escrow off and on, and to generate
multiple EA backup keys and keep them in safe locations.

\subsubsection{Users' Escrow Device Management}
\label{subsubsec:escrowmgmt}

When an account enables escrow and sets up an EA sigchain, all devices belonging to the account's
members will receive an unskippable notification that escrow is being enabled for their account, and
that the user's data will be available to the account's EAs. Clients will validate that escrow is
turned on by checking that the EA sigchain's \userID{} is in the account sigchain, and only accept
EA sigchains with lockdown mode enabled.

The notification shows the EA sigchain's fingerprint, which account admins should share with their
users through a secure out-of-band channel. Users should compare the fingerprint received from the
out-of-band channel with the one displayed by their clients. If the fingerprints differ, users
should reach out to their account admins to notify them of the potential MitM attack. Users will not
be able to use features leveraging their device keys until escrow is enabled.

Upon acknowledgement of the notification, clients generate a fresh set of device keypairs
corresponding to their new virtual escrow device, encrypt the secret keys for the EA sigchain's
latest PUK, delete the secret keys, and add the corresponding public keys with a
$\deviceaddandapprove{}$ link on their own sigchain. The server will prevent users from revoking
this virtual device (unless the administrators disable escrow).

If the EA sigchain's PUK rotates, the escrow device keys of all the users in the account (and their
PUKs) have to rotate as well, as all those keys are known to a revoked device. A
$\devicekeyrotate{}$ link that rotates an escrow device’s key can either be signed by any of the
users’ own devices or by one of the EA's device keys. At the moment, these rotations are performed
by each user’s own devices when they come online, but in the future we can have the EA’s devices
also perform them to ensure that the escrow device key rotation is completed in a timely manner,
especially for users who might not come online frequently. Escrowee clients do not display UI
notifications for these escrow device key rotations, because the system enforces that the only
devices that can rotate the escrow device keys are those on the user sigchain, and those trusted by
the device on the EA sigchain that enabled escrow (and that the user has acknowledged they trust).
This is achieved by having clients enforce that the EA sigchain is in lockdown mode, and including
the EA sigchain tail in the $\deviceaddandapprove{}$ introducing the escrow device. If later the
client detects that the EA sigchain is no longer in lockdown mode, or that the device which posted
the latest PUK on this chain is not trusted by the one which originally enabled lockdown mode (for
example, this could happen when lockdown mode is disabled and re-enabled by a different device on
that chain), it either throws an error or asks the user to acknowledge escrow again (with an updated
fingerprint).

The Zoom server immediately notifies escrowee clients to handle escrow admin's PUK rotations; in the
future, users will monitor the Zoom Transparency Tree for the EA and account sigchains, which would
prevent the server from withholding such updates. When learning that the EA or account sigchain have
been updated, for example before performing an escrow device key rotation as above, clients enforce
that escrow is still enabled and that the \userID{} of the EA sigchain mentioned has not changed (by
looking at the account sigchain), and that the EA sigchain has lockdown mode enabled. If clients
detect that escrow has been turned off, or that the EA \userID{} has changed, they will revoke their
escrow devices from their own chains, rotate their PUKs, and potentially notify the user and
re-enable escrow as above using the new \userID{}.

\subsubsection{EA Permissions}
\label{subsubsec:escrowadminperm}
While all the EAs' devices in the oldest class have access to the same PUK keys and can perform the
same actions from a cryptographic perspective, the Zoom servers allow assigning a more granular set
of permissions to each Escrow Administrator, which would apply to all their devices on the EA
sigchain. These additional restrictions align with typical business requirements our customers might
have, as detailed below, but could be circumvented if the Zoom server were to be compromised.

Users with the \texttt{escrow-manager} permission can enable escrow, create the first escrow device,
grant escrow permissions to other users, and approve new devices on the EA sigchain.

To support user recovery (see Section~\ref{subsubsec:userrecovery}) and legal discovery (see
Section~\ref{subsubsec:legaldiscovery}), we introduce two more server-enforced permissions. 

Users with the \texttt{escrow-write} permission (used for recovery) and the \texttt{escrow-read}
permission (used for discovery) both can add devices to the EA sigchain, but cannot approve the
device themselves (only users with the \texttt{escrow-manager} permission can approve new devices
for the EA). Therefore, to receive access to the EA PUKs, their newly added devices need to be
approved by a user with the \texttt{escrow-manager} permission. In addition, users with the
\texttt{escrow-write} permission can use their EA device keys to write approval links on other
account member's user sigchains (therefore helping them recover their data), while users with the
\texttt{escrow-read} permission can request from the server ciphertexts encrypted for any of the
account member in order to decrypt them.

\subsubsection{User Recovery}
\label{subsubsec:userrecovery}

In the case that a user in an account with escrow enabled loses access to their devices (including
any backup keys), they can ask one of their escrow account administrator for help recovering their
data on a newly logged-in device.

An EA with the \texttt{escrow-write} permission can use one of their authorized devices on the EA
sigchain to approve the user's new device and give them access to the necessary encryption keys. As
in a regular device approval (Section~\ref{subsubsec:devicemgmt}), the EA is first presented with an
approval review screen which includes all of the escrowee’s devices (and the escrowee's fingerprint,
which can be checked out-of-band with them). If the admin accepts, the admin's device signs a
$\batchapprove{}$ sigchain link on the escrowee's chain, and encrypts all of the user's previous
PUKs for the user's new device. The EA signs the approval link with both the virtual escrow device
key (from the user's sigchain) and the escrow admin's own device key (from the EA sigchain).

\subsubsection{Legal Discovery}
\label{subsubsec:legaldiscovery}

Some situations, such as legal discovery, will require the account admin to obtain data from its
account members.

If escrow is enabled, an EA with the \texttt{escrow-read} permission can request from the server
escrowees' data ciphertexts (such as emails encrypted with escrowee PUKs) and ciphertexts containing
the escrowee PUKs (encrypted for the virtual escrow device, the secret key of which is known to the
EA devices), and therefore decrypt the necessary data without cooperation from other account
members.
