\section{Introduction}
Hundreds of millions of participants join Zoom Meetings each day. They use Zoom to learn among classmates scattered by recent events, to connect with friends and family, to collaborate with colleagues and, in some cases, to discuss critical matters of state. Zoom users deserve excellent security guarantees, and Zoom is working to provide these protections in a transparent and peer-reviewed process. This document, mindful of practical constraints, proposes major security and privacy upgrades for Zoom.

We are actively engaging in a process of consultation with multiple stakeholders, including clients, cryptography experts, and civil society. As we receive feedback, we will update this document to reflect changes in roadmap and cryptographic design.

\subsection{Background and Current System}
Zoom Meetings currently use encryption to protect identity, data for meeting setup, and meeting contents. Zoom provides software for desktop and mobile operating systems and embeds software in Zoom Room devices. In this document, when we refer to ``Zoom clients'' we include all of these various forms of packaging. Crucially, these are systems to which we can deploy cryptographic software. Zoom Meetings also supports web browsers through a combination of WebRTC and custom code. If enabled, Zoom meetings support the use of clients not controlled by Zoom, namely phones using the public switched telephone network (PSTN) and room systems supporting SIP and H.323.

In the meeting setting (as opposed to webinars), Zoom supports up to 1,000 simultaneous users. When a Zoom client gains entry to a Zoom meeting, it gets a 256-bit per-meeting key created by Zoom's servers, which retain the key to distribute it to participants as they join. In the version of Zoom's meeting encryption protocol released on May 30, 2020, this per-meeting key is used to derive a per-stream key by combining the per-meeting key with a non-secret stream ID using an HMAC function. Each stream key is used to encrypt audio/video (UDP) packets using AES in GCM mode, with each client emitting one or more uniquely-identified streams. Those packets are relayed and multiplexed via one or more Multimedia Routers (MMR) in Zoom's infrastructure. The MMR servers do not decrypt these packets to route them. There is no mechanism to re-key a meeting.

Videoconferencing systems in which the server relies on plaintext access to the meeting content to perform operations such as multiplexing would be exceptionally difficult to secure end-to-end. In this design, we take advantage of the fact that the Zoom servers do not require any access to meeting content, allowing end-to-end security at exceptionally large scale. 

If a PSTN or SIP client is authorized to join, the MMR provides the per-meeting encryption key to specialized connector servers in Zoom's infrastructure. These servers act as a proxy: they decrypt and composite the meeting content streams in the same manner as a Zoom client and then re-encode the content in a manner appropriate for the connecting client. Zoom's optional Cloud Recording feature works similarly, recording the decrypted streams and hosting the resulting file in Zoom's cloud for the user to access. In the current design, Zoom's infrastructure brokers access to the meeting key.

This current design provides confidentiality and authenticity for all Zoom data streams, but it does not provide ``true'' end-to-end (E2E) encryption as understood by security experts due to the lack of end-to-end key management. In the current implementation, a passive adversary who can monitor Zoom's server infrastructure and who has access to the memory of the relevant Zoom servers may be able to defeat encryption. The adversary can observe the shared meeting key (MK), derive session keys, and decrypt all meeting data. Zoom's current setup, as well as virtually every other cloud product, relies on securing that infrastructure in order to achieve overall security; end-to-end encryption, using keys at the endpoints only, allows us to reduce reliance on the security of Zoom infrastructure.

\subsection{Goals and Threat Model}
This document proposes upgrades to Zoom that achieve end-to-end security against a range of powerful adversaries. In particular, we consider the following classes of adversaries:

\begin{description}
	\item {\bf Outsiders:} Individuals who are not part of Zoom's trusted infrastructure, and do not have access to non-public meeting access control information  (e.g., meeting passwords, IDs, SSO systems). These attackers may monitor, intercept, and modify network traffic, but do not have access to Zoom infrastructure.
	\item {\bf Meeting participants:} Participants who can access a meeting, because they know meeting's ID and password or exercise other qualifying credentials.
	\item {\bf Insiders:} Those who develop and maintain Zoom's server infrastructure and its cloud providers.    
\end{description}

Against these adversaries colluding or working independently, we seek the following security goals:

\begin{description}
	\item {\bf Confidentiality:} Only authorized meeting participants should have access to meeting audio and video streams. People removed from a meeting should have no further access after their expulsions.
	\item {\bf Integrity:} Those who are not allowed into a meeting should have no ability to corrupt the content of a meeting.
	\item {\bf Abuse Prevention:} When authorized meeting participants engage in abusive behavior, there is an effective mechanism to report them to Zoom's safety team, to help prevent further abuse.
\end{description}

We note that the current Zoom Meeting system has many highly-effective server-driven security mechanisms that are orthogonal to this proposal's concerns, and therefore remain unchanged.

\subsection{Limitations}
\label{subsec:limit}
To achieve these objectives would be an important improvement to Zoom's overall security and would give Zoom's users additional assurances that their meetings are secure along the axes they care most about. However, as with any security program, there are limitations to our approach.

First, we intend to leverage third-party Single Sign-Ons (SSOs) and Identity Providers ($\idp$s) to independently vouch for the identity of Zoom's users. Doing so moves the trust away from Zoom and to identity providers that many of Zoom's enterprise users already trust for sensitive identity operations. Where we do rely on SSOs and $\idp$s, meetings may become vulnerable because of attacks on their infrastructure.

Second, there are certain classes of attack and threats that we deem out of scope, including:

\begin{description}
	\item {\bf In-meeting impersonation attacks:} A malicious but otherwise authorized meeting participant colluding with a malicious server can masquerade as another authorized meeting participant.
	\item {\bf Metadata and traffic analysis:} Even for end-to-end encrypted meetings, insiders and outsiders can learn details about meeting duration, meeting bandwidth, data streaming patterns, participant lists and IP addresses.
	\item {\bf Software flaws:} Zoom's client code or the third-party libraries it links against can have bugs, or worse, intentional backdoors. Zoom's binary build procedures might become compromised. In these cases, there are no good guarantees we can make. Zoom relies on extensive analyses by independent third party auditors to reassure customers in this domain.
\end{description}

Third, we note that Zoom has a rich feature set and works on multiple platforms. Some of these features and use cases might be incompatible with strong cryptographic processes. Consider, for instance, dial-in phones or SIP/H.323 devices, which cannot be modified to support end-to-end encryption and require meeting content to be decrypted and re-encoded in an ``end'' in Zoom's data center. E2E security of the type contemplated by this paper is not possible in meetings that support these legacy standards. 

Fourth, users can access Zoom meetings through their web browser, and without installing Zoom's client. Supporting web users poses certain challenges: secure, long-term storage for cryptographic private keys might be unavailable; and worse, malicious web servers could feed backdoored source code to web users with little chance for discovery. We intend to participate in the web standards development process to facilitate the creation of browsers upon which we could offer dependable E2E security. 

Finally, the current document does not outline a solution for Zoom's webinar product, which supports meetings with up to 50,000 participants in a setting where many participate in a receive-only mode. Encryption for this setting will require a slightly different solution, and will be adapted from the current proposal once it is stable. Nor does this document outline a solution for the Zoom Chat product, which will require a different design that supports the asynchronous and persistent aspects of chat versus synchronous, ephemeral video conferencing.

\subsection{Outline}
This proposal lays out a long-term roadmap for E2E security in Zoom in four phases. The first phase is an upgrade of the meeting key exchange protocol to use public-key cryptography, hiding all secret keys from the server. The next three phases harden the notion of a user's identity---even across multiple devices---to help maintain server honesty in the key exchange and to give hosts better information when allowing or disallowing participation in a meeting. We provide the most detail for the earliest stages. We surmise that the specifics of later phases will change with more implementation and deployment experience.
