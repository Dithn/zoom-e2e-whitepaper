\section{Roadmap}
We propose a preliminary, incrementally-deployable four-phase roadmap.

\subsection{Phase I: Client Key Management}
In the first phase, we will roll out public key management, where every Zoom application generates and manages its own long-lived public/private key pairs; those private keys are known only to the client. From here, we will upgrade session key negotiation so that the clients can generate and exchange session keys without needing to trust the server. In this phase, a malicious party could still inject an unwanted public key into this exchange. We offer ``meeting security codes// as an advanced feature, so motivated users can verify public keys. The security to be achieved here will approximate those of Apple's FaceTime and iMessage products.

The key improvement in Phase I is that a server adversary must now become active (rather than passive) to break the protocol. In Phase I, we will support native Zoom clients and Zoom Rooms. We will not support Web browsers, PSTN dial-in, and other legacy devices. There also will be no support for ``Join Before Host'', Cloud Recording, and some other Zoom features.

\subsection{Phase II: Identity}
In the first phase, clients will trust Zoom to accurately map usernames to public keys. A malicious Zoom server in theory has the ability to swap mappings on-the-fly and to trick participants into entering a meeting with imposters. In Phase II, we will introduce two parallel mechanisms for users to track each other's identities without trusting Zoom's servers.  For users who authenticate to Zoom via Single-Sign-On (SSO), we will allow the SSO $\idp$ (Identity Provider) to sign a binding of a Zoom public key to an SSO identity, and to plumb this identity through to the UI. Unless the SSO or the $\idp$ has a flaw, Zoom cannot fake this identity. Second, we allow users to track contacts' keys across meetings. This way, the UI can surface warnings if a user joins a meeting with a new public key. 

\subsection{Phase III: Transparency Tree}
In the third phase, we will implement a mechanism that forces Zoom servers (and SSO providers) to sign and immutably store any keys that Zoom claims belong to a specific user, forcing Zoom to provide a consistent reply to all clients about these claims. Each client will periodically audit the keys that are being advertised for their own account and surface new additions to the user. Additionally, auditor systems can routinely verify and sound the alarm on any inconsistencies in their purview. In this scenario, if Zoom were to lie about Alice's keys (say, in order to join a meeting which Alice is invited to), it would have to lie to everyone in a detectable way. We will obtain these guarantees by building a ``transparency tree,'' similar to those used in Certificate Transparency~\cite{langley2013certificate} and Keybase~\cite{keybase}.

During this phase we will also provide the capability for meeting leaders to ``upgrade'' a meeting to end-to-end encrypted once it has begun, provided that all attendees are using the necessary client versions and incompatible features are not in use. Such incompatible features include PSTN dial-in, SIP/H.323 room systems and cloud recordings. Meetings that cannot be upgraded will have the option grayed-out. 

We also re-enable ``Join Before Host'' mode.

\subsection{Phase IV: Real-Time Security}
Consider this hypothetical attack against the Phase III design: a malicious Zoom server introduces a new ``ghost'' device for Bob, a user who does not have their IdP vouch for their identity. The attacker, using this fake new device, starts a meeting with Alice. Alice sees a new device for Bob but does not check the key fingerprint. After the fact, Bob can catch the server's malfeasance, but only after the attacker tricked Alice into divulging important information. The transparency tree encourages a ``trust but verify'' stance, where intrusions cannot be covered up. In Phase IV, we look to the future where Bob should sign new devices with existing devices, use an SSO $\idp$ to reinforce device additions, or delegate to his local IT manager. Until one of these conditions is met, Alice will look askance at Bob's new devices.