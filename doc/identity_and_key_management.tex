\label{sec:identitykeymanagement}
As part of an overarching end-to-end encryption solution, we introduce the concept of
\textit{cryptographic identity} to Zoom. This allows Zoom users to ensure they are communicating
only with the intended parties across many Zoom products, and helps prevent impersonation and
``meddler-in-the-middle attacks'' (MitM)\footnote{``Meddler-in-the-middle attack'' is also known as
``man-in-the-middle attack.''}. End-to-end encryption is only as secure as the ends: if Alice thinks
she is talking to her coworkers, but instead her competitors are participating in the meeting
(perhaps with their video off) or there is a MitM, encryption will not be
sufficient to protect her. Zoom aims to give users helpful and trustworthy information to evaluate
the identity of users they interact with.

This section describes how we define and represent a user's identity to the people they interact
with on Zoom, as well as how we enforce that these identities are consistent over time and cannot be
tampered with.

\subsection{Non-Cryptographic Identity at Zoom}

Zoom organizes its users into accounts. Accounts can be held by individual people, businesses or
institutions, and they consist of one or more users: if Example Corporation uses Zoom, then each
Example Corporation employee would be a Zoom user belonging to the Example Corporation account. Each
user can have more than one device (e.g., a computer or a phone) on which they can use Zoom
products.

Each account is part of a cloud infrastructure that hosts the data relating to the account and its
users, such as email addresses and login information. Some Zoom users are in the Zoom commercial
cloud; there is also a Zoom for Government cloud for U.S. government employees and contractors, as
well as separate white-labeled private Zoom instances, each with their own cloud. Zoom products
generally support cross-cloud communication, though there may be some limitations.

Zoom users authenticate to Zoom in a variety of ways. Users can log in using their email address and
a password, or via an OAuth or SAML-based flow with an external Identity Provider (\idp) that has
been set up for their account. In all of these cases, an email address is used as a unique user
identifier. If the account settings allow it, users can change their email address or authentication
method.

Some Zoom products, such as meetings, do not require individuals to sign in as a Zoom user in order
to participate, unless configured otherwise. Users can join a meeting by clicking a link or by
entering the meeting ID and password in the app.

The identity information displayed for a given user depends on the Zoom product, product-specific
settings, account settings, and whether the viewer is in the same account or a different account.
Identity information may include name, job title, company, phone number, and email address. Users
may be able to modify their identity information, though account administrators can restrict their
users to approved names. This identity information, and mechanisms that control changes to it, are
controlled by the Zoom servers and cannot be independently verified by clients.



Zoom products may provide mechanisms to enforce access control: for example, meetings support
meeting passwords, the waiting room feature, and the ability to restrict the meeting to users in the
host's account or users whose emails have a specific domain name. These features are enforced by the
Zoom servers, so they can be circumvented if the server is compromised. They also do not prevent one
member of an account from impersonating another member of the same account, and they may not give
users enough information to make access control decisions themselves: for example, whether to admit
an attendee from a meeting's waiting room.

\subsection{Cryptographic User Identity}

The cryptographic identity of a Zoom user consists of two components. The first component is a set
of human-readable identifiers unique to each user and the account they belong to. This allows users
to be identified by displaying their email addresses and information about their Zoom accounts to
other users. Second, each user's identity includes the set of devices (and their cryptographic keys)
controlled by that user. We describe a device model that lets us reason about how a user's devices
and keys change over time, and helps us formalize the concept of trust between devices. The device
model allows a user's devices to communicate amongst themselves in addition to
securely communicating with other users. When a device, logged in for a given
user, is first used for a feature requiring cryptographic identity, it
automatically generates encryption and signing key pairs, a process called
\textit{provisioning}. These device key pairs will then be used to
negotiate additional encryption key pairs shared between the users' devices.

The components of a user's identity can change over time, and it is important to keep track of these
changes so that they are auditable. For this purpose, we introduce a data structure called a signed
hashchain, or \textit{sigchain}.

Finally, the device model is reflected in the user interface. Users are notified when new devices
are added, and are able to revoke devices that are lost, stolen, or no longer used.

\subsection{Displaying Identity}
\label{subsec:displayid}

\textbf{Note:} Displaying cloud identifiers and ADNs is not currently available. We plan to release
them in future updates.

This section describes how we could display the identity of a Zoom user to others in various
products. Because cryptographic keys are not easy to read, compare, or keep track of, we only show
human-readable identifiers in the user interface. A user's set of identifiers consists of three
components:

\begin{enumerate}
\item A Cloud Identifier, which represents the cloud infrastructure a user's information is hosted
    on (omitted if the user is on the Zoom commercial cloud)
\item An Account Domain Name (ADN), which identifies the account that the user is part of, where
    applicable
\item
    An email address, which can be used to distinguish individual users within the account
\end{enumerate}

Here are a few examples of how a user's identifiers can be displayed to another user:

\begin{multicols}{2}
John Smith \\
\texttt{example.com} (\texttt{jsmith@example.net})

\columnbreak

The display name, ``John Smith," is freely chosen and not authenticated. \texttt{example.com} is the
ADN\@. Note that the email domain, \texttt{example.net}, can differ from the ADN\@.
\end{multicols}

\begin{multicols}{2}
Lucy Lee \\
\texttt{example.org} (\texttt{lucy.lee@example.org}) \\ \textbf{GOV}

\columnbreak

Since the \texttt{example.org} company works with the US government, their identities and keys are
hosted on the separate Zoom for Government Cloud, and this is noted in the UI.
\end{multicols}

\begin{multicols}{2}
Anna Smith \\
\texttt{example.com}

\columnbreak

Anna might decide not to disclose her email address but still be identified as a member of the
\texttt{example.com} account. In this case, although the user's email address would not be revealed,
their devices' long-term cryptographic public keys could be leveraged by a determined attacker to
ascertain when they have interacted with the same device multiple times, even when the display name
is altered.
\end{multicols}

\clearpage

\begin{multicols}{2}
Richard Roe

\columnbreak

Users can use some Zoom products as guests and display no identifying information to other users,
other than the freely-chosen display name. Since they generate fresh long-term keys each time, the
aforementioned tracing attack is not possible.
\end{multicols}

\begin{multicols}{2}
Mike Doe \\
(\texttt{mike.doe@example.com})

\columnbreak

Mike is associated with an email address but not an ADN.
\end{multicols}

\subsubsection{Identifying Accounts}\label{subsubsec:adn}

\textbf{Note:} ADN support is not currently available. We plan to release it in a future update.

Accounts on Zoom can be optionally identified using a domain name, which we call the Account Domain
Name (ADN). Domain names make good identifiers because they are unique (while for example two
companies with the same name might exist in two countries), and many users are already familiar with
them. We will allow internationalized domain names (IDNs) to be used as ADNs, but to prevent
homograph attacks, the UI will show the Punycode representation by default and the rendered domain
name only on mouse hover.

An account administrator may choose one of the account's existing Associated Domains as the ADN, or
Zoom can provide a new dedicated subdomain. Associated Domains is an existing Zoom feature where a
Zoom account can be associated with multiple domain names. To do so, the account administrator has
to prove ownership of the domain name to Zoom, e.g., by adding a specific DNS record, adding a
header tag to the home page, or hosting a file at a specific location on the domain.

Only a single account at a time can use a domain name as their ADN. Account administrators are
allowed to change their ADN if needed.

\subsubsection{Identifying Users}

Zoom users are also identified by email address. Emails (unlike names) are unique, and they
sometimes represent people better than the legal name held in a company's HR system. Most users
already have email addresses they are associated with, such as the one they use to log in. Zoom Mail
Service (see Section~\ref{sec:email}) also issues users a dedicated email address upon signup. For
Zoom Mail Service users, the email address used as identifier will be their Zoom Mail Service
address. We may support associating multiple email addresses per user in the future.

In some cases, users will be able to change the email associated with themselves. When a user Alice
changes their email from \texttt{support@example.com} to \texttt{alice@example.com}, for example,
other users that interacted with Alice are not notified of this change. However, if a new user Bob
takes over Alice's old email \texttt{support@example.com} and associates it with his existing user,
then people who interacted with \texttt{support@example.com} when it was associated with Alice will
get a prompt explaining that \texttt{support@example.com} is now associated with a new user. These
notifications are supported by the Contact Sync feature (planned for a future release); see
Section~\ref{subsec:contactsync}.

\subsection{Multi-Device Support}
\label{subsec:multidev}

Zoom users often have several devices that they use Zoom on: their work computer, personal computer,
mobile phone, and so on. Each of these devices stores long-term signing and encryption keys (called
\textit{device keys}), which it uses to secure communications and data across a variety of Zoom
products. To obtain a rigorous concept of identity, we formalize the set of a user's device keys as
well as the ways this set can change over time, a crucial step towards achieving the goal of linking
a user's identifiers with their device keys.

Users have three main operations to change their set of valid devices:
\begin{enumerate}
\item $\deviceadd{}$, which adds a new device to the set. The new device generates a new long-term
    signing key, which is used directly in some Zoom products (such as E2EE meetings) and also to
    sign statements about the set, and an encryption key, which can be used to communicate with
    other devices.
\item $\devicerevoke{}$, which revokes the validity of a previously-added device. Revoked devices
    are still recorded as part of the set for auditing purposes, but new signatures from their keys
    are no longer considered valid, and the encryption keys they have access to are rotated as soon
    as possible.
\item $\batchapprove{}$, which indicates that an existing device considers all devices added to the
    set until this $\batchapprove{}$ event legitimate and trustworthy. This operation is signed by
    the approving device's key.
\end{enumerate}

Ensuring that each user has a single set of devices that is consistent over time serves several
purposes. For the user themselves, it ensures that all of their devices know about each other, so
they can be notified when a new device gets added and quickly react if their user account has been
compromised.

A user's set of devices is also of interest to other users they interact with, because not all
devices associated with a user may be trusted equally. If Bob is in a meeting with Alice's work
computer today, he can trust the connection (i.e., that there is no MitM) by either checking the
security code or by noticing that Alice's public key is the same as was used in all past
interactions Bob had with Alice. This way, Bob only has to trust that there was no MitM the first
time the connection was established, and from that assumption deduce that all communications where
Alice has the same public key are also secure. This assumption is commonly referred to as
Trust-On-First-Use (TOFU). If tomorrow, Bob meets with Alice's new mobile phone, Bob might not trust
its key as much as her work computer's: a malicious server could have added it, or a hacker could
have stolen Alice's Zoom password. It'd be unfortunate if Alice and Bob had to recheck their
security code. By performing a $\batchapprove{}$ operation from her work computer, Alice can
indicate that all of her other devices are trusted, so Bob (who trusts the public key on Alice's
work laptop) can use the signed $\batchapprove{}$ statement to extend his trust to Alice's new
mobile phone's key.

$\batchapprove{}$ links induce a trust graph over a user's set of devices, where each device
represents a node and each $\batchapprove{}$ adds an edge from the device performing the approval to
all non-revoked devices introduced after that device. We call each connected component in this graph
an \textit{approval class}, and we assume devices in the same approval class trust each other. When
a Zoom user provisions a new device, they'll have access to their complete device list and so will
be able to revoke any that are unrecognized, lost, or stolen. Because of this, we assume that later
devices implicitly trust the validity of earlier ones.

Consider the following scenario:
\begin{enumerate}
\item Bob provisions Device $\zdev{a}$
\item Bob provisions Device $\zdev{b}$
\item Bob provisions Device $\zdev{c}$
\end{enumerate}

Bob's graph is disconnected: he has 3 separate devices and 3 different approval classes. $\zdev{c}$
does implicitly trust $\zdev{b}$, but we don't consider them part of the same approval class as the
trust is not mutual.

\begin{enumerate}
  \setcounter{enumi}{3}
\item Bob logs onto $\zdev{b}$ and performs a $\batchapprove{}$
\end{enumerate}

This operation partially connects Bob's trust graph. $\zdev{b}$ trusts $\zdev{c}$ (the device
provisioned after $\zdev{b}$). Now, there are only two approval classes: one with $\zdev{a}$ and one
with $\zdev{b}$ and $\zdev{c}$.

\begin{enumerate}
  \setcounter{enumi}{4}
\item Bob provisions Device $\zdev{d}$
\item Bob logs onto $\zdev{c}$ and performs a $\batchapprove{}$
\end{enumerate}

Now, $\zdev{c}$ trusts $\zdev{d}$. Because $\zdev{b}$ already trusted $\zdev{c}$, we know that
$\zdev{b}$ trusts $\zdev{d}$ as well, even though it never made this claim explicitly.

\begin{enumerate}
  \setcounter{enumi}{6}
\item A malicious server provisions Device $\zdev{e}$ in order to impersonate Bob
\item Bob logs onto $\zdev{a}$, revokes $\zdev{e}$ since it's unrecognized, and performs a
    $\batchapprove{}$
\end{enumerate}

Having all of Bob's devices know about each other allows Bob to take action if his account is
compromised. After he revokes $\zdev{e}$, all of Bob's devices, as well as the users he communicates
with on Zoom, know not to trust statements signed by $\zdev{e}$'s device key.
Figure~\ref{fig:devices} summarizes Bob's trust graph after these steps.

\begin{figure}
\centering
\begin{tikzpicture}[->,>=stealth',shorten >=1pt,auto,node distance=3.4cm,semithick]
  \node[shape=circle,fill=blue,text=white,draw=none]         (A)              {$\zdev{a}$};
  \node[shape=circle,fill=blue,text=white,draw=none]         (B) [right of=A] {$\zdev{b}$};
  \node[shape=circle,fill=blue,text=white,draw=none]         (C) [right of=B] {$\zdev{c}$};
  \node[shape=circle,fill=blue,text=white,draw=none]         (D) [right of=C] {$\zdev{d}$};
  \node[shape=circle,fill=red,text=white,draw=none]          (E) [right of=D] {$\zdev{e}$};

  \path (A) edge                node {step 8} (B) edge [bend left]    node {step 8} (C) edge [bend
            right]   node {step 8} (D) (B) edge                node {step 4} (C) (C) edge
            node {step 6} (D);
\end{tikzpicture}
\caption{Device approval graph, where nodes are devices and edges are approvals.}
\label{fig:devices}
\end{figure}

\subsubsection{Per-User Keys}\label{subsubsec:puks} One application of the trust graph is to
facilitate \textit{per-user keys} (PUKs): a set of keys shared between all of a user's devices,
rotated on device addition or revocation.

In each $\deviceadd{}$ or $\devicerevoke{}$ operation, devices generate a new PUK seed. The PUK seed
is encrypted for each unrevoked device's device key: devices implicitly trust all older devices
(otherwise, they would revoke them). Each seed is associated with a number called a \textit{PUK
generation}, which starts at 1 and increments every time the seed rotates. All devices within an
approval class share the same set of known PUK seeds.

From each PUK seed, devices generate subseeds, symmetric subkeys, and asymmetric subkeys for various
applications such as email and voicemail.

Symmetric PUKs facilitate syncing encrypted data between a user's devices. Devices use the latest
per-user key to encrypt all content, but previous per-user keys are still useful for decrypting
older data. In the above example, if $\zdev{f}$ is provisioned, it doesn't yet have access to older
PUKs; only the one it just created. But if an older device performs a $\batchapprove{}$ that
includes $\zdev{f}$, it will also encrypt all the PUK seeds it knows about for the devices it's
approving, which means $\zdev{f}$ can now decrypt data encrypted with keys created before $\zdev{f}$
was introduced.

Asymmetric encryption PUKs can be used to encrypt data for other users. If Alice encrypts a piece of
content for Bob using his most recent asymmetric PUK, Bob can read the content on all devices added
before the content was encrypted, as well as all the devices added afterwards, as long as those
devices have been approved by an earlier one. The public keys corresponding to the asymmetric keys
derived from the PUK seed are published in the user's sigchain
(Section~\ref{subsubsec:usersigchains}).

To generate a set of PUKs, devices:

\begin{enumerate}
\item
Generate a new 32-byte secret seed
\item
Use HKDF on the seed to generate different 32-byte keys:
\begin{enumerate}
\item
private X25519, context \texttt{"Zoombase-2-ClientOnly-KDF-PerUserX25519"}
\item
email seed, context \texttt{"Zoombase-2-ClientOnly-KDF-PerUserEmailSeed"}
\begin{enumerate}
\item
  private X25519 for email, context \\ \texttt{"Zoombase-2-ClientOnly-KDF-PerUserEmailX25519"}
\end{enumerate}
\item
voicemail seed, context \texttt{"Zoombase-2-ClientOnly-KDF-PerUserVoicemailSeed"}
\begin{enumerate}
\item
  private X25519 for voicemail, context  \\
  \texttt{"Zoombase-2-ClientOnly-KDF-PerUserVoicemailX25519"}
\end{enumerate}
\item
symmetric, context \texttt{"Zoombase-2-ClientOnly-KDF-PerUserSymmetricKey"}
\end{enumerate}
\end{enumerate}


See Appendix~\ref{appendix:multidev} for a deeper analysis of multi-device configurations and the
guarantees we can achieve given these rules. In summary, a newly-added device will immediately be
able to send and receive new encrypted content that is accessible from all devices. Approving a
device will give that device access to past encrypted content. If a device is revoked, that device
is unable to decrypt any content encrypted after the revocation.

Note that if a device is self-revoked, the necessary key rotation might not happen immediately if
there are no other suitable devices online to perform it. Before the key rotation is completed, we
rely on the server to enforce that the revoked device does not access further user data, including
ciphertexts sent both before and after the revocation, by no longer accepting the per-device
authentication tokens and signatures using the keys the device owns. This is not a cryptographic
guarantee and relies on the server's honesty, and also does not cover any content already in the
device's memory before revocation.

\subsection{Consistent Identities With Sigchains}
\label{subsec:sigchains}

Both accounts and users have states that change over time. An account can change its identity
provider or its ADN, and a user can change their email address or add and remove devices.

We need to keep track of states that change over time in a way that is auditable. To do so, we
describe the sequence of changes in a data structure called a signed hashchain, or
\textit{sigchain}.

Once a client learns of a sigchain, the only changes to this chain that will be considered valid are
extensions of the sequence. Since changes cannot be ``forgotten,'' the Zoom server cannot rewrite
history.

Still, this model doesn't force the Zoom server to be consistent across different devices it talks
to. We will add a transparency layer called the Zoom Transparency Tree (Section~\ref{sec:ztt}) to
ensure that the Zoom server must present the same information about sigchains to all users.

\subsubsection{Sigchains}

A sigchain is a sequence of statements (called \textit{links}), where each link includes a
collision-resistant hash of the previous link. These links can be thought of as state transitions
that modify an object (the sigchain state). For a user sigchain, the sigchain state contains the
list of active devices, the list of revoked devices, the trust graph, and the list of email
addresses and accounts historically associated with the user.

In order to accept a transition as valid, clients check that it satisfies several conditions,
including that:

\begin{enumerate}
\item The link is of a known type.
\item The link has the correct fields for that type.
\item The transition is admissible given the current state.
\item The link correctly includes the hash of the previous link.
\item Some links require cryptographic signatures by the devices authorizing the transition to be
    considered valid. In these cases, the signatures are encoded as part of the links to compute
    link hashes.
\end{enumerate}

Examples of admissibility rules for a user sigchain include that a device can only be revoked if it
was active in the previous state, and that signatures over revocation links must be by a device that
was active in the previous state.

Since each of the links in a sigchain contains a hash of the previous link, the hash of the last
link is a compact commitment to the entire sigchain state. Each sigchain link also contains an
incrementing sequence number. We refer to an object consisting of the sigchain type, the last link's
sequence number, and the last link's hash as the \textit{sigchain tail}:

\begin{Verbatim}
{
  "sigchainType": "User",
  "lastSequenceNumber": 15,
  "lastLinkHash": "484ad7..."
}
\end{Verbatim}

When clients want to update their view of a sigchain from the server, they can just query for the
new links and ensure that the first new link contains a previous hash matching the cached tail.

Note that the examples of objects in this document are encoded in JSON and simplified for ease of
exposition. The actual implementation may use different application encodings and data structures.

Different applications require different levels of access to sigchains. For example, although a user
should be able to fully audit the history of past email addresses stored in their sigchain, meeting
participants might only need to see the most recent one to display it in the UI\@. For this reason,
rather than being directly encoded, sensitive information on a sigchain link is stored as a
commitment: rather than \texttt{alice@example.com}, a sigchain link could contain
$$\mathsf{COMMIT}(\texttt{alice@example.com}) =
    \mathsf{HMAC}(\mathsf{randomKey}, \texttt{alice@example.com}).$$ %
The server gives all users the entire sigchain link so that they can check that its signatures
and hashes are valid, but only Alice's devices will receive the plaintext email addresses and
32-byte random keys corresponding to previous email addresses. \textsf{COMMIT} can also be used
to selectively delete parts of links such as device names: a server can throw away the random
HMAC key as well as the plaintext data, and the signature over the link will still verify.

\subsubsection{Overview of Sigchain Types}

Zoom devices, users, and accounts are each internally identified by unique immutable identifiers
called \deviceid, \userID, and \accountID respectively. The \deviceid is generated randomly for each
new device, while the \userID and \accountID are assigned by the server.

Each device, user, and account is also associated with more user-friendly (but mutable) identifiers:
respectively, device names, email addresses, and ADNs.

Representing these different components and their relationships requires different types of
sigchains:

\begin{enumerate}
\item User sigchains store, for each \userID, information related to that user's identity, such as
    the user's email address, \accountID, and the set of their devices and their trust
    relationships.
\item Email sigchains store, for each email address, the associated \userID.
\item Account sigchains store, for each \accountID, both the ADN and identity provider associated with
    the account.
\item ADN sigchains keep track of, for each domain name, the \accountID to which the domain is
    associated.
\item Membership sigchains keep track of, for each \accountID, the {\userID}s within the account.
\end{enumerate}

Note that some of the information stored on these sigchains is redundant: for example, a mapping
between an email and the corresponding \userID is recorded both in a user sigchain and in an email
sigchain. This is necessary so that the server cannot claim that two separate {\userID}s are
associated with the same email address at the same time. Accordingly, some operations will cause
multiple chains to be updated at the same time.

As detailed earlier, every Zoom user is part of an account. As potential optimizations, if this
account only has a single user and doesn't have an ADN, then that user's sigchain might not mention
their \accountID, and there would be no corresponding account or membership sigchains until the
account either gets another user or an ADN.

\subsubsection{User Sigchains}
 \label{subsubsec:usersigchains}
User sigchains record changes to a user's identity. There are several types of user sigchain links,
each representing a different way to change a user's identity.

\paragraph{EmailChange.} As mentioned in Section~\ref{subsec:displayid}, users can set and change the
email addresses that will be displayed for or used to communicate with other users. Two users can switch email addresses,
but the server will prove that two users do not have the same email address at the same time. An
\textsf{EmailChange} link will have the following fields:


\begin{Verbatim}
{
  "sigchainType": "User",
  "linkType": "EmailChange",
  "sequenceNumber": 10,
  "prev": "484ad7...",
  "cloudName": "commercial",

  "userID": "ebc0d2...",

  "emailChange": COMMIT({
    "email": "alice@example.com",
    "emailChainSequenceNumber": 5
  })
}
\end{Verbatim}

The first six fields are common to every sigchain link. \textsf{sequenceNumber} is an incrementing
counter that starts from 1 for the first link, \textsf{prev} is the (canonical) hash of the previous
link in the chain (in this case, the one with sequence number 9), and \textsf{cloudName} specifies
which cloud the sigchain belongs to.

Every user sigchain link also specifies the \userID.

Here, the \textsf{email} field specifies the new email address to be associated with this user,
which supersedes any previous email. Every time an \textsf{EmailChange} link associates this user
with a new email, the email sigchain for that email address is also extended with a corresponding
\textsf{UserIDChange} link referring to this user's \userID, and the sequence number of that link is
reported in this link as \textsf{emailChainSequenceNumber}.

Because the Zoom website can be used to change one's email address, \textsf{EmailChange} links do
not have any signatures (i.e., they can be inserted into sigchains by the Zoom server).

\paragraph{AccountChange.} Users can also transfer between accounts, similarly to how they can
switch emails.

\begin{Verbatim}
{
  "sigchainType": "User",
  "linkType": "AccountChange",
  ...
  "accountChange": COMMIT({
    "accountID": "c2d8aa...",
    "membershipChainSequenceNumber": 12,
    "additionIndex": 5
  })
}
\end{Verbatim}

Since multiple users can be added to a membership sigchain in a single link, \textsf{additionIndex}
specifies the corresponding position in that link. If a user is removed from an account, the
\textsf{accountChange} section specifies \accountID \texttt{null} and the other two fields are
omitted.

Since Zoom servers are allowed to move users between accounts, \textsf{AccountChange} links do not
have any signatures.

\paragraph{DeviceAdd.} A device addition link specifies the long-term public device keys for the new
device and a human-readable name:

\begin{Verbatim}
{
  "sigchainType": "User",
  "linkType": "DeviceAdd",
  ...
  "deviceID": "ebc0d2...",
  "deviceName": COMMIT({
    "name": "Alice's Work Smartphone",
    "version": 1
   }),
  "deviceType": "PHONE",
  "deviceEd25519PublicKey": "ce8564...",
  "deviceX25519PublicKey": "ad7913...",

  "userEmailX25519PublicKey": "f40b66...",
  "userVoicemailX25519PublicKey": "d600fc...",
  "userX25519PublicKey": "c2cce1...",

  "revokeDeviceIDs": [
     "ac98ad...",
     ...
  ]
}
\end{Verbatim}

This link specifies a device identifier (unique among the devices associated with this user), a
signing public key, an encryption public key, as well as a device name and type. Note that the
device name is hidden by a commitment, because users other than the owner of the chain do not need
to see it. In order to support reusing names, the device name includes an incrementing version
component, which will be visible in the user interface. Device names allow the user to have a
human-readable, unambiguous way to distinguish their devices.

The link also specifies the new per-user key public keys. The encryptions of the new PUK seeds for
older devices are not represented within the sigchain link, but are sent separately to the server,
which propagates them to the older devices.

After the user adds a new device, they can immediately use it to revoke any existing undesired
devices, which justifies our claim that later devices should trust the validity of earlier
(unrevoked) ones.

To ensure that users know the corresponding device private key, each $\deviceadd{}$ link requires a
signature by \textsf{deviceEd25519PublicKey}.

\paragraph{UserRoot.} This link should always be the first link in a user's sigchain, and can only
appear as such. It is equivalent to a DeviceAdd and specifies all the same kinds of information
about the user's first device, but can also include \textsf{emailChange} and \textsf{accountChange}
fields, which convey similar information as the corresponding sigchain links. Including these fields
here allows us to reduce the overall number of sigchain links.

\paragraph{DeviceRename.} Users can change their device names if desired. This change is signed with
the device's public key.

\paragraph{DeviceRevoke.} When a device is stolen, lost, or no longer used for Zoom, the user should
revoke it. If one of the user's valid devices performs the revocation, it will also rotate the PUKs
and sign this link to guarantee integrity of the new key. If instead the revocation is done through
the Zoom website, the PUK cannot be rotated and no signature is made. After a device signing key is
revoked, it can no longer be used to sign sigchain links.

\paragraph{DeviceKeyRotate.} If a user suspects their device was temporarily compromised, or if they
have institutional key rotation policies, they might want to rotate their device keys and PUKs. This
operation keeps the same \deviceid but chooses new signing and encryption keys, as well as a new set
of PUKs, and uploads encryptions of each PUK secret known to the device for the new encryption key
so that the old encryption key can be safely discarded. This link is signed by the device's previous
public key in addition to the new key. As with $\devicerevoke{}$ links, the old signing device key
cannot sign further sigchain links.

\paragraph{BatchApprove.} A $\batchapprove{}$ link lets a device indicate that it trusts the
validity of all devices created after that device until the point in the sigchain where the
$\batchapprove{}$ link appears. To reduce the number of sigchain links, the user can also specify a
list of devices to revoke within this link. The link can specify a new set of PUKs, and this is
required if any devices are being revoked. This link is signed by the approving device, and can be
used to construct a trust graph as described in Section~\ref{subsec:multidev}.

\paragraph{DeviceAddAndApprove.} This link is equivalent to a fused $\deviceadd{}$ and
$\batchapprove{}$, but allows us to reduce the overall number of sigchain links and also guarantee
that intermediate states of some operations can't appear on the sigchain. It is posted by an
existing device, the approver, but also specifies a device identifier, name, and keys for a new
device. Uniqueness of the new device's identifier, name, and keys are enforced just like in
$\deviceadd{}$. This link implies that the approver trusts all devices created up to and including
the device being added by this link. Signatures from both the approver and the new device are
required.

\paragraph{PerUserKeyRotate.} If a device notices that a device revocation (either a self
revocation, or one from the website) has occurred after the most recent PUK was generated, the device
will perform another PUK rotation using a $\peruserkeyrotate{}$ link. This guarantees that even if
the revoked device was compromised, this newest PUK is still confidential. For personal storage,
staleness is not an issue, as any device trying to encrypt data will first rotate the per-user keys.
But if other users are encrypting data for a compromised PUK and the server cooperates, data could
be readable by a revoked device. This link is signed by the device that is rotating the PUK.

\subsubsection{Email Sigchains}

Users can change their email over time. It is important that at any specific point in time, each
email corresponds to a unique user, so that Zoom users can be unequivocally identified. Also, users
should be able to audit whether at any point their email has been associated with another user. We
record the mapping between an email and the corresponding {\userID}s in an email sigchain. These
sigchains only have one kind of link, \textsf{UserIDChange}.

\begin{Verbatim}
{
  "sigchainType": "Email",
  "linkType": "UserIDChange",
  ...
  "email": COMMIT("alice@example.com"),

  "userIDChange": COMMIT({
    "userID": "ebc03d...",
    "userChainSequenceNumber": 9
   })
}
\end{Verbatim}

The \textsf{userChainSequenceNumber} refers to the position in the user sigchain of the
corresponding \textsf{EmailChange} link (or the initial \textsf{UserRoot} link).

Since users can change their email on the Zoom website, this link does not require any signatures.

\subsubsection{Account Sigchains}

Account sigchains consists of two kinds of links, which record the identity provider and ADN that
each account is using. \textsf{IDPUpdate} links contain the domain name of the IDP (say
\texttt{examplecorp.generic-idp.com}). IDPs should not use the same domain for multiple accounts
to prevent equivocation of their attestations.

\begin{Verbatim}
{
  "sigchainType": "Account",
  "linkType": "IDPUpdate",
  ...
  "accountID": "abef02...",

  "idpDomain": "examplecorp.generic-idp.com"
}
\end{Verbatim}

\textsf{ADNChange} links associate an ADN with the account. Only the latest \textsf{ADNChange} link
is considered valid, and each one corresponds to an \textsf{AccountIDChange} link in the appropriate
ADN sigchain. If an account is no longer using an ADN, the \textsf{adnChange} section specifies ADN
\texttt{null} and the other field is omitted.

\begin{Verbatim}
{
  "sigchainType": "Account",
  "linkType": "ADNChange",
  ...
  "accountID": "abef02...",

  "adnChange": COMMIT({
    "adn": "example.org",
    "adnChainSequenceNumber": 9
  })
}
\end{Verbatim}

Links in account sigchains do not require any signatures.

\subsubsection{ADN Sigchains}

ADN sigchains track which \accountID a specific Account Domain Name is associated with. There is
only one type of link, \textsf{AccountIDChange}, which corresponds to and points to an
\textsf{ADNChange} link in the account sigchain for the appropriate \accountID:

\begin{Verbatim}
{
  "sigchainType": "ADN",
  "linkType": "AccountIDChange",
  ...
  "adn": "example.org",

  "accountIDChange": COMMIT({
    "accountID": "873c34...",
    "accountChainSequenceNumber": 29
  })
}
\end{Verbatim}

Links in ADN sigchains do not require any signatures.

\subsubsection{Membership Sigchains}

Membership sigchains record changes to the set of users which are part of each account. They have a
single type of link, \textsf{ChangeMembers}:

\begin{Verbatim}
{
  "sigchainType": "Membership",
  "linkType": "ChangeMembers",
  ...
  "accountID": "19aebb...",

  "added": [
    COMMIT({
      "userID": "db2f1c...",
      "userChainSequenceNumber": 9,
    }),
    ...
  ],
  "removed": [
    COMMIT({
      "userID": "9ae3d2...",
      "userChainSequenceNumber": 4
    }),
    ...
  ]
}
\end{Verbatim}

Each link can add or remove multiple users. Each user is hidden behind a commitment so that it is
possible to prove that an individual user is part of an account without also leaking the {\userID}s
of the other members that are being added as part of the same link. Hiding the {\userID}s of the
users being removed from the account would potentially make it harder to prove that a user is indeed
still a member of a specific account without opening all the commitments. We will solve this problem
in the future by leveraging the transparency layer, but until then, clients will not rely on these
sigchains, although the server will start keeping track of them.

Links in membership sigchains do not require any signatures.

\subsection{Sigchain Fingerprints}
\label{subsec:fingerprints}
A sigchain fingerprint summarizes a user's sigchains. It is constructed as a hash of the user's user
sigchain and account sigchain tails (note that this includes the user's present and past email
addresses, as well as the current ADN due to the reference in the account sigchain). Sigchain
fingerprints are useful in two contexts: to verify new devices before approving, and to verify the
keys used to communicate with other users.

During the device addition and approval process, a malicious insider has the opportunity to conduct
a MitM attack. After the new device is added, the insider creates a copy of the
user's sigchain but replaces the keys in the final $\deviceadd{}$ link with keys controlled by the
attacker, using the same device name. The forged copy is served to older devices while the honest
sigchain is served to the new device. When the older device sees the approval interface, only the
device name is shown, and so the user may be tricked into accepting the server's device
unintentionally (which would imply sharing all previously encrypted data with the attacker).
However, in this case the two devices would necessarily display different fingerprints (as they bind
to sigchains containing different keys, and the hash is collision resistant). Users concerned about
MitM attacks can check, before approving new devices, that the fingerprint shown on
the approving device matches the one from the device(s) they intend to approve. This ensures that an
attacker cannot interfere with the process and gain access to historical encrypted data.

Sigchain fingerprints are also useful when communicating with other users. For example, when a user
emails another user for the first time, a malicious server could potentially swap out the intended
user's sigchain for an entirely fabricated sigchain with devices controlled by the server. In these
contexts, the user interface will allow users to verify the other user's sigchain fingerprint before
sending the encrypted message. This process requires the two users to securely communicate
out-of-band (for example, in-person). The sender can check that the fingerprint they see in their
client matches the recipient's fingerprint to ensure the message can be decrypted only by the
intended recipient. Similarly, the recipient can do the same for the sender to verify that the
message came from the expected sender and was not altered.

In the user interface, fingerprints are represented in hexadecimal. Users can always view their own
fingerprint from the Zoom client, and Zoom products will show the fingerprints of other users in
specific scenarios where it is useful.

\subsection{Backup Keys}
\label{subsec:backupkeys}

A backup key acts as a ``virtual device'' that can be added to a user's sigchain. Instead of
corresponding to an actual device, it is simply represented as a string of letters and digits, such
as
\[ \texttt{ZR30 4D11 5HJM RJG2 6H75 78DH B0VS 4KSF}. \] The first four characters are used as a key
identifier and are not considered private. Backup keys have at least 128 bits of private entropy, as
well as built-in error correction to tolerate small copy-pasting mistakes.

From a trusted device's Zoom client, a user can generate a backup key and add it as a new ``device''
to the user's sigchain with a $\deviceaddandapprove{}$ link. The backup key string is used to derive
a seed using sodium{}'s~\cite{libsodium} \texttt{argon2id}, which is then used to derive the
device's signing and encryption keys. Like normal devices, backup keys can be revoked.


\subsubsection{Account Recovery}

In the case that a user loses all their devices, they can log onto another device and enter their
backup key to write a $\batchapprove{}$ link to the user's sigchain that originates from the backup
key. This gives the new device access to all the old PUKs that the backup key had access to.

\subsubsection{Using Backup Keys for Escrow}\label{subsec:backupkeyescrow}

\textbf{Note:} Escrow using backup keys is not currently available. Users can still
manually generate backup keys and send them to their IT administrator out of band.

Organizations using Zoom may wish to have access to E2E communications involving their members, to
protect against data loss or to support legal retention and discovery. We refer to solutions that
give such organizations cryptographic access as \textit{escrow}.  While we are building a more
usable and robust escrow solution, backup keys can initially provide the minimal necessary
functionality: users can generate backup keys and send them to their IT department for safekeeping.
When escrow is enabled for an account, users in that account are notified before they are allowed to
continue use of Zoom products. Each user receives notice that their data is being escrowed the
first time they access their Zoom Mail Service email on one of their devices. The user should verify
the escrow administrator's sigchain fingerprint before continuing, after which the user's client
automatically sends an end-to-end encrypted email with a backup key to their account administrator.


\subsection{Client Key Management}
\label{subsec:clientkeys}

\subsubsection{Storing Secret Keys on Device}
\label{subsec:lks}

Clients persist device key pairs indefinitely until a $\devicerevoke{}$ or $\devicekeyrotate{}$
occurs. Device keys are never transmitted to any other device or the server. They may sometimes be
lost after a disk corruption or operating system reinstall. In this case, the user must go through
the provisioning process once again as a new device would.

Long-term device keys are stored in the local operating system's keychain (where available), but
with some added protection for two Zoom users using the same OS account.
%
Before storing keys in the local OS-provided keychain, we encrypt them using a key-wrapping key
stored by the server and specific to each user and each device.
%
When a user revokes their device, they will delete the keychain entry and the server will delete the
corresponding key. This also guarantees that keys cannot be recovered from a backup of a device that
has since been revoked.
%
If two users are using the same computer, the key-wrapping key prevents one user from being able to
access the other's keys.

We use the committing AEAD scheme $\mathsf{CtE1}$~\cite{messagefranking} to prevent the server from
supplying malicious key-wrapping keys.
%
On provisioning, after generating the device encryption and signing key pairs, the client will
encrypt each secret key $k$ as follows:
\begingroup
\RaggedRight
\begin{enumerate*}
\item Generate a 32-byte random string $\mathsf{KWK}$ and request the server to store it
persistently associated with the user and device.
\item Define $\context \leftarrow \texttt{"Zoombase-1-ClientOnly-KDF-SecretStore"}$.
\item Compute $C \leftarrow \textsf{CtE1-Enc}(\textsf{K}{=}\textsf{KWK}, \textsf{H}{=}\context,
\textsf{M}{=}k)$, where $\textsf{H}$ is the associated data parameter for the underlying AEAD, and
store it in the system keychain.
\end{enumerate*}
\endgroup

For \textsf{CtE1}, we use \HMACSHATWO as the commitment function and
\sodium{}~\cite{libsodium}'s \linebreak \texttt{crypto\_aead\_chacha20poly1305\_ietf} as the AEAD.

We emphasize that this feature will not protect against an insider who also has access to a user's
device (for example, by colluding with another user using the same device). It will also not prevent
different users of the same device from installing malware to steal the other user's keys.

\subsubsection{Device Management Interface}

We offer a dedicated UI to manage devices that are part of the user's sigchain, available in the
Zoom client's settings. Upon visiting this device list, clients ask the Zoom server for the latest
sigchain tail and process any new links in order to make sure that the view is up-to-date. The
device list contains all active devices (which can be used to participate in E2EE communications)
and revoked devices (which can no longer be used), indicating their device name and type based on
the sigchain;\footnote{The list may also include some information about devices that is not directly
stored on the sigchain and that we rely on the server to report honestly. For example, it may
include the earliest time of any E2E-encrypted communication decryptable by each device, determined
as the time at which the oldest device in its approval class was added.} it also has the user's own
fingerprint and the user's current and past email addresses (if any). Users can revoke devices from
this view. If a device realizes that it is revoked (by processing updates to its own sigchain, for
example as part of a periodic refresh or because of a server notification), it will delete all
private ephemeral and long-term keys as well as sensitive data, and then log itself out.

When the user first uses a feature that requires sigchains from one of their devices, that device
generates a new set of device keys and adds them to the first link in the user's sigchain. From then
on, the user's other devices (existing and future) will also generate their own keys and extend the
chain; each time, the user is prompted to review the device list and revoke any devices that are
unrecognized, lost, stolen, or no longer used.

In addition, after provisioning each new device, the user gets notifications on their old devices
asking them to approve or revoke any new untrusted devices. This list might include devices that are
already revoked but are still new from the perspective of the old device. Users also get
notifications regarding changes made to their email address.

Once the Zoom Transparency Tree (Section~\ref{sec:ztt}) is deployed, Zoom servers and insiders will
not have the ability to, e.g., suppress notifications in order to hide a malicious device addition
or email change.

\subsection{Highlighting Untrusted Devices with Contact Sync}
\label{subsec:contactsync}

\textbf{Note:} Contact Sync is not currently available. We plan to release it in a future update.
The description here is specific to E2EE Meetings (Section~\ref{sec:meetings}), but similar
guarantees can be extended to other products. 

Even with a strong concept of identity, impersonation attacks are still possible in meetings. If
Bob's coworker, Alice, has the email \texttt{alice@company.org}, Bob might not realize if he joins a
meeting with an impersonator using \texttt{al1ce@company.org}, especially if the impersonator has
their video turned off. Or if a hacker stole Alice's username and password, they could provision a
new device and pretend to be Alice. We'd like to have a TOFU-style UI feature to let Bob know if
it's the first time he's seeing a device in a meeting---but we also don't want to bother Bob for his
first meeting with Alice on every new device that Alice makes, which could lead to alert fatigue.
Specifically, we want to provide warnings when Bob is in a meeting with a device that is not
approved by a device that Bob has seen before. We'd also like Bob to share his meeting history
between his devices securely so they all have the latest information.

Each device maintains, for every other user that it has been in a meeting with, a record containing
that user's user sigchain tail, the time of their first meeting together, and the total number of
E2EE meetings they were in together. These records are updated after each meeting as appropriate,
though they are only updated when either
\begin{enumerate}
\item The meeting has less than 25 participants and the device has been in the meeting with the
participant for over 10 minutes
\item The participant has been speaking for over 30 seconds
\end{enumerate}

In order for the meeting history to be shared across devices, clients periodically send encrypted
meeting records to the server. Records are individually encrypted with the latest PUK, signed with
the device key, and tagged with $t = \mathsf{HMAC}(k, \userID)$, where $k$ is a key derived from the
PUK. The server stores a mapping between $(\deviceid, t)$ and the encrypted record, which the
clients can update as necessary. When the client learns of a new PUK, records and tags are updated
to use the new PUK lazily.

The records are encrypted to minimize the privacy loss in case they are leaked (Zoom servers already
learn who is participating in which meeting). Because the tag is generated deterministically, these
records would reveal if two of Alice's devices had meetings with the same user (but not which user),
given that both devices used the same per-user key. However, we find this tradeoff acceptable as it
allows for a more efficient synchronization. Further, the server has the ability to rollback a
device's record at any time, but doing so could only cause additional warnings.

When joining a meeting, clients generate tags for each other participant using each known PUK and
request the server to send any corresponding records. Any record signed by a revoked device is not
considered. Given this data, we can provide warnings for each device in a meeting:
\begin{enumerate}
\item If the device has been seen before, or if the device is trusted by a device that has been seen
    before as indicated by the trust graph, display the number of meetings with this user in the
    last month.
\item Otherwise, if the device is untrusted but the user has been seen before, display ``This is the
    first time you are talking to this device of this person."
\item Otherwise, display ``This is the first time you are talking to this person."
\end{enumerate}

Note that devices may not have access to all the per-user keys used to encrypt records, which could
result in extraneous warnings. For example, imagine Alice provisions Zoom on her phone and has a
meeting with Bob. Then, Alice provisions Zoom on her laptop and has another meeting with Bob without
approving the laptop from the phone. In this case, Alice's laptop does not have the PUK used by
Alice's phone to encrypt the record, so there will be another warning for Bob in the second meeting.

Though Contact Sync warnings improve Zoom users' security, alert fatigue can make the warnings less
useful. If a user's contacts regularly fail to approve new devices with earlier ones, or if the
server does not properly sync records of prior communication, users can grow accustomed to the
warnings and possibly ignore them in the event of a real attack. Even without an attack, excessive
warnings make for a suboptimal and tiresome user experience.

\subsection{Compromise Prevention for Device Provisioning}\label{subsec:realtime} \textbf{Note: }
The features in this section are not currently available. We plan to release them in future updates.

With the ZTT (see Section~\ref{sec:ztt}), we have a mechanism for users to detect sigchain
corruptions after-the-fact. For instance, if an insider adds a new device onto Alice's sigchain to
impersonate her in a meeting while she is offline, Alice will detect the attack after she comes back
online, since the attacker had to commit the malicious change to Alice's sigchain to the ZTT.
Ideally, Alice could prevent unauthorized device additions from happening in the first place. Note
that users whose identities are vouched for by a trusted third-party IDP
(Section~\ref{sec:idpattestations}) already have good protections as long as their IDP is not
compromised. We want to offer a strong alternative to all users that does not rely on external
parties.

Recall that users currently login to Zoom via SSO, OAuth, or simple username and password. Their
devices' keys are authenticated by multiple signatures: a self signature, an optional approval
signature by an older device (Section~\ref{subsec:multidev}), a signature by Zoom (as part of the
inclusion in the ZTT, once it is deployed), and an optional attestation from the IDP.

Further validations of the device keys are possible. For example, users can sign new devices with
existing devices at the time they are first added (as opposed to in a later approval), potentially
through a QR-code-based flow similar to the one Keybase has now. One device shows a QR code of a
random session key, and the other scans the QR code; this establishes an end-to-end secure tunnel
between the two devices that a compromised server cannot interfere with. Over this tunnel, devices
can exchange public keys and cross-sign each other.

Alternatively, organizations can set up policies that require an IT administrator to sign off on
device additions. The new user device can display the hash of its public key in QR code form. The IT
administrator can scan it, signing the new device with their administrator's signing key.

Devices without the required signatures might be prevented from signing on behalf of the user or
rotating the user's PUKs, thus denying them access to any content encrypted for the user. They might
also trigger extra UI warnings to other participants when joining meetings, or even be excluded from
meetings marked ``high security.''


\subsection{Security Properties}
\label{sec:IdKmProps}

The identity and key management system described in this section is leveraged by multiple Zoom
products, and it provides some lower level security properties that these products build upon.

First, we note that the secret keys corresponding to any device public keys included in any sigchain
are known only to the device that generated them (unless an attacker has somehow gained access to
the device's memory or storage): secret keys never leave the device,\footnote{Except possibly when
the user voluntarily sends crash reports to Zoom. We try to minimize this risk, but cannot exclude
it.} and are only used to perform encryption and signing. A consequence is that
ciphertexts/signatures for/by each of the device keys can only be decrypted/created by the device
that generated these keys. An exception is backup keys, which are shown once to the user as text
strings that can be written down and sometimes sent to the user's account administrators as an
escrow mechanism (after the user receives notice). In this case, our guarantees depend on these
keys being kept securely by the user and their administrator, and not being available to an
attacker.

Similarly, consider any valid user sigchain. The secret keys corresponding to any per-user public
key that appears in that sigchain are only known to any devices that were added but not revoked
before the per-user key was added to that sigchain, plus any devices that were added afterwards and
approved by one of those devices (as recorded in each device's own view of the user's sigchain).

Also, note that the Zoom server cannot force any devices to forget identity updates like device
revocations: when receiving sigchains, devices only accept new sigchain links that extend the ones
they are already aware of.

\subsubsection{MitM Between a User's Devices}
These properties allow us to obtain strong guarantees about the confidentiality of the data that is
encrypted for a user's device keys and PUKs. For example, consider any ciphertext (such as an email
draft) encrypted for a given per-user key whose secret key is only known to a set of uncompromised
devices controlled by the corresponding user. In order to obtain the plaintext, an adversary (even
an insider) would have to either break the encryption scheme, compromise one of the user's current
devices or backup keys, or trick the user into using one of their devices to
approve a maliciously-controlled device (so that the existing device encrypts
the PUK secrets for the malicious one). A cautious user would only approve an
additional device on their sigchain in a situation where they
expect such a request: for example, after they sign in on Zoom for the first time on a new device.
An insider could take this as an opportunity to attempt a MitM attack, by lying about the new
device's public key to the old device. However, outsiders cannot perform this attack: even if the
attacker had access to the user's credentials, the attacker would have to add an extra additional
device and cannot replace the old one from the approval modal. Comparing the sigchain fingerprints
as known to the approving device and the new device before confirming approval also prevents
insiders from performing this attack. Because the sigchain is an append-only data structure,
comparing fingerprints after approval would still allow the user to detect the attack, providing
evidence of server compromise/misbehavior and allowing the user to mitigate its effects.

In the future, the Zoom Transparency Tree will offer (assuming trusted auditors) an automated and transparent way to
ensure that devices have a consistent view of all sigchains and public keys such that checking
fingerprints before device approvals will no longer be critical.

\subsubsection{MitM Between Different Users}
Another class of attacks involves an MitM between the sender and recipient of an encrypted
communication. An insider might lie to the sender about the recipient's sigchain and PUKs or add a
malicious device to the recipient's sigchain, causing the sender to encrypt for a key controlled by
the attacker. These attacks can be prevented or detected by checking fingerprints, and can in many cases
be detected after-the-fact with minimal user burden by leveraging the ZTT.

Consider an adversary compromising one of the recipient's devices or backup keys, thus obtaining their
secrets. With access to the device, an attacker could obtain all of the recipient's data, both locally cached
and stored by the server (at least until their session expires). If the user discovers the
compromise, they could revoke this device and rotate their keys, preventing further data leakage.
However, if the attacker also compromises (even at a later point) the Zoom server infrastructure,
they might attempt a more subtle attack: when a sender requests this recipient's sigchain, the
server could provide a legitimate but stale version of this chain, i.e., one that does not include
the device revocation and the subsequent rotation. This attack can be prevented by
comparing fingerprints. However, detecting this compromise using fingerprints is not that
straightforward: the sender and recipient would have to ensure that they agree on their views of the recipient's sigchain at the time the \emph{message was encrypted and sent}, rather than at the time of the
fingerprint comparison. Assuming all clients have synchronized clocks, one way to achieve this would be
to have each sigchain link include a timestamp. Each client would remember the last time
they updated their view of any sigchain and refuse to accept new links for that sigchain that have an
earlier timestamp. In order to be tolerant to time misconfigurations, we currently do not enforce
these properties, but we are considering them for future updates.

Eventually, the ZTT might also ensure, under similar time synchronization assumptions, that
everyone's view of all sigchains is not only consistent, but also relatively up-to-date, i.e.\ that
any attempt to withhold updates beyond some reasonable tolerance bound is detected. This will ensure
that the server is not able to trick senders into encrypting for out-of-date keys unnoticed.

\subsubsection{Integrity}
Ensuring that all devices have a consistent view of a user's sigchain and that no
extraneous devices have been added to it (by comparing fingerprints or
relying on the ZTT and monitoring one's own sigchain) also helps with integrity
guarantees. Users can ensure communications haven't been tampered with by
checking that they are signed by devices belonging to the claimed author. However, as above, in some
cases the evidence of compromise might be less conclusive: if the sender signs a message using a
device's signing key, and later revokes that device (for example, because the device was lost or
compromised), the recipient has no way to tell if the message was signed before or after the
revocation/compromise. The user interface may communicate this potential risk to the recipient so
that they can confirm the integrity of any sensitive communications with the sender out-of-band, or
ask them to resend the message with an up-to-date key.

\subsubsection{Security Limitations}

While our concept of identity provides strong security guarantees, there are still attacks it won't
be able to prevent.

An attacker who is able to register a new device in the system on behalf of a non-consenting user,
whether by stealing the user's credentials or compelling the server, will be able to read data that
is E2E-encrypted to the targeted user after the compromise (but not before, as explained above), as
it will be able to add a new compromised PUK to the user's chain, which might be used by other users
to encrypt messages intended for the recipient before the recipient has had a chance to come online
to review and revoke the new device. This attack can also be prevented if the sending user compares
fingerprints out-of-band with the potentially compromised recipient(s) before encrypting for the new
PUKs, or detected by comparing fingerprints (possibly implicitly through the ZTT) after the fact.

When one of a user's devices is revoked, the per-user public encryption keys for that user become
stale and should be rotated as soon as possible. When another of the user's devices comes back
online, they will pick new encryption key pairs, publish the public keys in a new sigchain link, and
encrypt the secret portions for all the user's devices that are still active. If one of the user's
devices is revoking another one, key rotation can happen immediately. However, in case of
self-revocation or revocation from the web or by an account administrator, if none of the other
user's devices are online, this rotation might be delayed. Any data encrypted to this user during
this period would be using an encryption key known to a revoked device, and we rely on the server to
enforce that the revoked device can no longer access user data. An attacker who both compromised a
revoked device \textit{and} had read access to the server might thus be able to decrypt this data.
Although the attack window seems limited and hard to exploit in practice, this is a limitation of
the current design. Similarly, when a key is revoked and rotated out, we do not re-encrypt old
ciphertexts for the new key, and rely on the server for access control to the ciphertexts.

\subsubsection{Privacy Limitations}

From a privacy perspective, in addition to the identifiers displayed in the user interface, our
solution provides some limited extra information about a user to other users they interact with: the
sigchains reveal the history of the user's devices, including when they were added and revoked (but
not their names, which are protected behind a commitment), as well as which device was used to sign
any specific E2E-encrypted communication. Similarly, the number of times that a user changes their
email address or account is visible (but not the previous emails or account IDs). Moreover, since
the server might report the timestamps of sigchain statements (and, once deployed, the ZTT will also
necessarily reveal the same information), time correlations between different statements might be
exploited to infer, for example, that two users swapped their email addresses. While this
information is not displayed in the user interface, the client needs this data to perform the
sigchain validation and therefore a motivated attacker might extract such information. We believe
that this is acceptable; it is similar to the security code change warnings in applications like
Signal and WhatsApp.

Note that sigchains are not publicly available and are subject to server-side access control: they
are provided as needed to users. For example, if Bob is a Zoom Mail Service user and knows Alice's
email address, they can ask the server for the sigchain associated with that address in order to
encrypt an email. We currently rate limit requests for users' sigchains and other personal data as a
partial mitigation for this leakage.

In the future, we are considering a different mechanism that allows the server to offer
``randomized'' versions of a user's encryption keys, in a way that trades the opportunity to check a
user's fingerprints (while still being able to detect impersonation after the fact) for increased
privacy.
