\subsection{Highlighting Untrusted Devices with Contact Sync}
\label{subsec:contactsync}

\textbf{Note:} Contact Sync is not currently available. We plan to release it in a future update.
The description here is specific to E2EE Meetings (Section~\ref{sec:meetings}), but similar
guarantees can be extended to other products. 

Even with a strong concept of identity, impersonation attacks are still possible in meetings. If
Bob's coworker, Alice, has the email \texttt{alice@company.org}, Bob might not realize if he joins a
meeting with an impersonator using \texttt{al1ce@company.org}, especially if the impersonator has
their video turned off. Or if a hacker stole Alice's username and password, they could provision a
new device and pretend to be Alice. We'd like to have a TOFU-style UI feature to let Bob know if
it's the first time he's seeing a device in a meeting---but we also don't want to bother Bob for his
first meeting with Alice on every new device that Alice makes, which could lead to alert fatigue.
Specifically, we want to provide warnings when Bob is in a meeting with a device that is not
approved by a device that Bob has seen before. We'd also like Bob to share his meeting history
between his devices securely so they all have the latest information.

Each device maintains, for every other user that it has been in a meeting with, a record containing
that user's user sigchain tail, the time of their first meeting together, and the total number of
E2EE meetings they were in together. These records are updated after each meeting as appropriate,
though they are only updated when either
\begin{enumerate}
\item The meeting has less than 25 participants and the device has been in the meeting with the
participant for over 10 minutes
\item The participant has been speaking for over 30 seconds
\end{enumerate}

In order for the meeting history to be shared across devices, clients periodically send encrypted
meeting records to the server. Records are individually encrypted with the latest PUK, signed with
the device key, and tagged with $t = \mathsf{HMAC}(k, \userID)$, where $k$ is a key derived from the
PUK. The server stores a mapping between $(\deviceid, t)$ and the encrypted record, which the
clients can update as necessary. When the client learns of a new PUK, records and tags are updated
to use the new PUK lazily.

The records are encrypted to minimize the privacy loss in case they are leaked (Zoom servers already
learn who is participating in which meeting). Because the tag is generated deterministically, these
records would reveal if two of Alice's devices had meetings with the same user (but not which user),
given that both devices used the same per-user key. However, we find this tradeoff acceptable as it
allows for a more efficient synchronization. Further, the server has the ability to rollback a
device's record at any time, but doing so could only cause additional warnings.

When joining a meeting, clients generate tags for each other participant using each known PUK and
request the server to send any corresponding records. Any record signed by a revoked device is not
considered. Given this data, we can provide warnings for each device in a meeting:
\begin{enumerate}
\item If the device has been seen before, or if the device is trusted by a device that has been seen
    before as indicated by the trust graph, display the number of meetings with this user in the
    last month.
\item Otherwise, if the device is untrusted but the user has been seen before, display ``This is the
    first time you are talking to this device of this person."
\item Otherwise, display ``This is the first time you are talking to this person."
\end{enumerate}

Note that devices may not have access to all the per-user keys used to encrypt records, which could
result in extraneous warnings. For example, imagine Alice provisions Zoom on her phone and has a
meeting with Bob. Then, Alice provisions Zoom on her laptop and has another meeting with Bob without
approving the laptop from the phone. In this case, Alice's laptop does not have the PUK used by
Alice's phone to encrypt the record, so there will be another warning for Bob in the second meeting.

Though Contact Sync warnings improve Zoom users' security, alert fatigue can make the warnings less
useful. If a user's contacts regularly fail to approve new devices with earlier ones, or if the
server does not properly sync records of prior communication, users can grow accustomed to the
warnings and possibly ignore them in the event of a real attack. Even without an attack, excessive
warnings make for a suboptimal and tiresome user experience.
