\subsection{Consistent Identities With Sigchains}
\label{subsec:sigchains}

Both accounts and users have states that change over time. An account can change its identity
provider or its ADN, and a user can change their email address or add and remove devices.

We need to keep track of states that change over time in a way that is auditable. To do so, we
describe the sequence of changes in a data structure called a signed hashchain, or
\textit{sigchain}.

Once a client learns of a sigchain, the only changes to this chain that will be considered valid are
extensions of the sequence. Since changes cannot be ``forgotten,'' the Zoom server cannot rewrite
history.

Still, this model doesn't force the Zoom server to be consistent across different devices it talks
to. We will add a transparency layer called the Zoom Transparency Tree (Section~\ref{sec:ztt}) to
ensure that the Zoom server must present the same information about sigchains to all users.

\subsubsection{Sigchains}

A sigchain is a sequence of statements (called \textit{links}), where each link includes a
collision-resistant hash of the previous link. These links can be thought of as state transitions
that modify an object (the sigchain state). For a user sigchain, the sigchain state contains the
list of active devices, the list of revoked devices, the trust graph, and the list of email
addresses and accounts historically associated with the user.

In order to accept a transition as valid, clients check that it satisfies several conditions,
including that:

\begin{enumerate}
\item The link is of a known type.
\item The link has the correct fields for that type.
\item The transition is admissible given the current state.
\item The link correctly includes the hash of the previous link.
\item Some links require cryptographic signatures by the devices authorizing the transition to be
    considered valid. In these cases, the signatures are encoded as part of the links to compute
    link hashes.
\end{enumerate}

Examples of admissibility rules for a user sigchain include that a device can only be revoked if it
was active in the previous state, and that signatures over revocation links must be by a device that
was active in the previous state.

Since each of the links in a sigchain contains a hash of the previous link, the hash of the last
link is a compact commitment to the entire sigchain state. Each sigchain link also contains an
incrementing sequence number. We refer to an object consisting of the sigchain type, the last link's
sequence number, and the last link's hash as the \textit{sigchain tail}:

\begin{Verbatim}
{
  "sigchainType": "User",
  "lastSequenceNumber": 15,
  "lastLinkHash": "484ad7..."
}
\end{Verbatim}

When clients want to update their view of a sigchain from the server, they can just query for the
new links and ensure that the first new link contains a previous hash matching the cached tail.

Note that the examples of objects in this document are encoded in JSON and simplified for ease of
exposition. The actual implementation may use different application encodings and data structures.

Different applications require different levels of access to sigchains. For example, although a user
should be able to fully audit the history of past email addresses stored in their sigchain, meeting
participants might only need to see the most recent one to display it in the UI\@. For this reason,
rather than being directly encoded, sensitive information on a sigchain link is stored as a
commitment: rather than \texttt{alice@example.com}, a sigchain link could contain
$$\mathsf{COMMIT}(\texttt{alice@example.com}) =
    \mathsf{HMAC}(\mathsf{randomKey}, \texttt{alice@example.com}).$$ %
The server gives all users the entire sigchain link so that they can check that its signatures
and hashes are valid, but only Alice's devices will receive the plaintext email addresses and
32-byte random keys corresponding to previous email addresses. \textsf{COMMIT} can also be used
to selectively delete parts of links such as device names: a server can throw away the random
HMAC key as well as the plaintext data, and the signature over the link will still verify.

\subsubsection{Overview of Sigchain Types}

Zoom devices, users, and accounts are each internally identified by unique immutable identifiers
called \deviceid, \userID, and \accountID respectively. The \deviceid is generated randomly for each
new device, while the \userID and \accountID are assigned by the server.

Each device, user, and account is also associated with more user-friendly (but mutable) identifiers:
respectively, device names, email addresses, and ADNs.

Representing these different components and their relationships requires different types of
sigchains:

\begin{enumerate}
\item User sigchains store, for each \userID, information related to that user's identity, such as
    the user's email address, \accountID, and the set of their devices and their trust
    relationships.
\item Email sigchains store, for each email address, the associated \userID.
\item Account sigchains store, for each \accountID, both the ADN and identity provider associated with
    the account.
\item ADN sigchains keep track of, for each domain name, the \accountID to which the domain is
    associated.
\item Membership sigchains keep track of, for each \accountID, the {\userID}s within the account.
\end{enumerate}

Note that some of the information stored on these sigchains is redundant: for example, a mapping
between an email and the corresponding \userID is recorded both in a user sigchain and in an email
sigchain. This is necessary so that the server cannot claim that two separate {\userID}s are
associated with the same email address at the same time. Accordingly, some operations will cause
multiple chains to be updated at the same time.

As detailed earlier, every Zoom user is part of an account. As potential optimizations, if this
account only has a single user and doesn't have an ADN, then that user's sigchain might not mention
their \accountID, and there would be no corresponding account or membership sigchains until the
account either gets another user or an ADN.

\subsubsection{User Sigchains}
 \label{subsubsec:usersigchains}
User sigchains record changes to a user's identity. There are several types of user sigchain links,
each representing a different way to change a user's identity.

\paragraph{EmailChange.} As mentioned in Section~\ref{subsec:displayid}, users can set and change the
email addresses that will be displayed for or used to communicate with other users. Two users can switch email addresses,
but the server will prove that two users do not have the same email address at the same time. An
\textsf{EmailChange} link will have the following fields:


\begin{Verbatim}
{
  "sigchainType": "User",
  "linkType": "EmailChange",
  "sequenceNumber": 10,
  "prev": "484ad7...",
  "cloudName": "commercial",

  "userID": "ebc0d2...",

  "emailChange": COMMIT({
    "email": "alice@example.com",
    "emailChainSequenceNumber": 5
  })
}
\end{Verbatim}

The first six fields are common to every sigchain link. \textsf{sequenceNumber} is an incrementing
counter that starts from 1 for the first link, \textsf{prev} is the (canonical) hash of the previous
link in the chain (in this case, the one with sequence number 9), and \textsf{cloudName} specifies
which cloud the sigchain belongs to.

Every user sigchain link also specifies the \userID.

Here, the \textsf{email} field specifies the new email address to be associated with this user,
which supersedes any previous email. Every time an \textsf{EmailChange} link associates this user
with a new email, the email sigchain for that email address is also extended with a corresponding
\textsf{UserIDChange} link referring to this user's \userID, and the sequence number of that link is
reported in this link as \textsf{emailChainSequenceNumber}.

Because the Zoom website can be used to change one's email address, \textsf{EmailChange} links do
not have any signatures (i.e., they can be inserted into sigchains by the Zoom server).

\paragraph{AccountChange.} Users can also transfer between accounts, similarly to how they can
switch emails.

\begin{Verbatim}
{
  "sigchainType": "User",
  "linkType": "AccountChange",
  ...
  "accountChange": COMMIT({
    "accountID": "c2d8aa...",
    "membershipChainSequenceNumber": 12,
    "additionIndex": 5
  })
}
\end{Verbatim}

Since multiple users can be added to a membership sigchain in a single link, \textsf{additionIndex}
specifies the corresponding position in that link. If a user is removed from an account, the
\textsf{accountChange} section specifies \accountID \texttt{null} and the other two fields are
omitted.

Since Zoom servers are allowed to move users between accounts, \textsf{AccountChange} links do not
have any signatures.

\paragraph{DeviceAdd.} A device addition link specifies the long-term public device keys for the new
device and a human-readable name:

\begin{Verbatim}
{
  "sigchainType": "User",
  "linkType": "DeviceAdd",
  ...
  "deviceID": "ebc0d2...",
  "deviceName": COMMIT({
    "name": "Alice's Work Smartphone",
    "version": 1
   }),
  "deviceType": "PHONE",
  "deviceEd25519PublicKey": "ce8564...",
  "deviceX25519PublicKey": "ad7913...",

  "userEmailX25519PublicKey": "f40b66...",
  "userVoicemailX25519PublicKey": "d600fc...",
  "userX25519PublicKey": "c2cce1...",

  "revokeDeviceIDs": [
     "ac98ad...",
     ...
  ]
}
\end{Verbatim}

This link specifies a device identifier (unique among the devices associated with this user), a
signing public key, an encryption public key, as well as a device name and type. Note that the
device name is hidden by a commitment, because users other than the owner of the chain do not need
to see it. In order to support reusing names, the device name includes an incrementing version
component, which will be visible in the user interface. Device names allow the user to have a
human-readable, unambiguous way to distinguish their devices.

The link also specifies the new per-user key public keys. The encryptions of the new PUK seeds for
older devices are not represented within the sigchain link, but are sent separately to the server,
which propagates them to the older devices.

After the user adds a new device, they can immediately use it to revoke any existing undesired
devices, which justifies our claim that later devices should trust the validity of earlier
(unrevoked) ones.

To ensure that users know the corresponding device private key, each $\deviceadd{}$ link requires a
signature by \textsf{deviceEd25519PublicKey}.

\paragraph{UserRoot.} This link should always be the first link in a user's sigchain, and can only
appear as such. It is equivalent to a DeviceAdd and specifies all the same kinds of information
about the user's first device, but can also include \textsf{emailChange} and \textsf{accountChange}
fields, which convey similar information as the corresponding sigchain links. Including these fields
here allows us to reduce the overall number of sigchain links.

\paragraph{DeviceRename.} Users can change their device names if desired. This change is signed with
the device's public key.

\paragraph{DeviceRevoke.} When a device is stolen, lost, or no longer used for Zoom, the user should
revoke it. If one of the user's valid devices performs the revocation, it will also rotate the PUKs
and sign this link to guarantee integrity of the new key. If instead the revocation is done through
the Zoom website, the PUK cannot be rotated and no signature is made. After a device signing key is
revoked, it can no longer be used to sign sigchain links.

\paragraph{DeviceKeyRotate.} If a user suspects their device was temporarily compromised, or if they
have institutional key rotation policies, they might want to rotate their device keys and PUKs. This
operation keeps the same \deviceid but chooses new signing and encryption keys, as well as a new set
of PUKs, and uploads encryptions of each PUK secret known to the device for the new encryption key
so that the old encryption key can be safely discarded. This link is signed by the device's previous
public key in addition to the new key. As with $\devicerevoke{}$ links, the old signing device key
cannot sign further sigchain links.

\paragraph{BatchApprove.} A $\batchapprove{}$ link lets a device indicate that it trusts the
validity of all devices created after that device until the point in the sigchain where the
$\batchapprove{}$ link appears. To reduce the number of sigchain links, the user can also specify a
list of devices to revoke within this link. The link can specify a new set of PUKs, and this is
required if any devices are being revoked. This link is signed by the approving device, and can be
used to construct a trust graph as described in Section~\ref{subsec:multidev}.

\paragraph{DeviceAddAndApprove.} This link is equivalent to a fused $\deviceadd{}$ and
$\batchapprove{}$, but allows us to reduce the overall number of sigchain links and also guarantee
that intermediate states of some operations can't appear on the sigchain. It is posted by an
existing device, the approver, but also specifies a device identifier, name, and keys for a new
device. Uniqueness of the new device's identifier, name, and keys are enforced just like in
$\deviceadd{}$. This link implies that the approver trusts all devices created up to and including
the device being added by this link. Signatures from both the approver and the new device are
required.

\paragraph{PerUserKeyRotate.} If a device notices that a device revocation (either a self
revocation, or one from the website) has occurred after the most recent PUK was generated, the device
will perform another PUK rotation using a $\peruserkeyrotate{}$ link. This guarantees that even if
the revoked device was compromised, this newest PUK is still confidential. For personal storage,
staleness is not an issue, as any device trying to encrypt data will first rotate the per-user keys.
But if other users are encrypting data for a compromised PUK and the server cooperates, data could
be readable by a revoked device. This link is signed by the device that is rotating the PUK.

\subsubsection{Email Sigchains}

Users can change their email over time. It is important that at any specific point in time, each
email corresponds to a unique user, so that Zoom users can be unequivocally identified. Also, users
should be able to audit whether at any point their email has been associated with another user. We
record the mapping between an email and the corresponding {\userID}s in an email sigchain. These
sigchains only have one kind of link, \textsf{UserIDChange}.

\begin{Verbatim}
{
  "sigchainType": "Email",
  "linkType": "UserIDChange",
  ...
  "email": COMMIT("alice@example.com"),

  "userIDChange": COMMIT({
    "userID": "ebc03d...",
    "userChainSequenceNumber": 9
   })
}
\end{Verbatim}

The \textsf{userChainSequenceNumber} refers to the position in the user sigchain of the
corresponding \textsf{EmailChange} link (or the initial \textsf{UserRoot} link).

Since users can change their email on the Zoom website, this link does not require any signatures.

\subsubsection{Account Sigchains}

Account sigchains consists of two kinds of links, which record the identity provider and ADN that
each account is using. \textsf{IDPUpdate} links contain the domain name of the IDP (say
\texttt{examplecorp.generic-idp.com}). IDPs should not use the same domain for multiple accounts
to prevent equivocation of their attestations.

\begin{Verbatim}
{
  "sigchainType": "Account",
  "linkType": "IDPUpdate",
  ...
  "accountID": "abef02...",

  "idpDomain": "examplecorp.generic-idp.com"
}
\end{Verbatim}

\textsf{ADNChange} links associate an ADN with the account. Only the latest \textsf{ADNChange} link
is considered valid, and each one corresponds to an \textsf{AccountIDChange} link in the appropriate
ADN sigchain. If an account is no longer using an ADN, the \textsf{adnChange} section specifies ADN
\texttt{null} and the other field is omitted.

\begin{Verbatim}
{
  "sigchainType": "Account",
  "linkType": "ADNChange",
  ...
  "accountID": "abef02...",

  "adnChange": COMMIT({
    "adn": "example.org",
    "adnChainSequenceNumber": 9
  })
}
\end{Verbatim}

Links in account sigchains do not require any signatures.

\subsubsection{ADN Sigchains}

ADN sigchains track which \accountID a specific Account Domain Name is associated with. There is
only one type of link, \textsf{AccountIDChange}, which corresponds to and points to an
\textsf{ADNChange} link in the account sigchain for the appropriate \accountID:

\begin{Verbatim}
{
  "sigchainType": "ADN",
  "linkType": "AccountIDChange",
  ...
  "adn": "example.org",

  "accountIDChange": COMMIT({
    "accountID": "873c34...",
    "accountChainSequenceNumber": 29
  })
}
\end{Verbatim}

Links in ADN sigchains do not require any signatures.

\subsubsection{Membership Sigchains}

Membership sigchains record changes to the set of users which are part of each account. They have a
single type of link, \textsf{ChangeMembers}:

\begin{Verbatim}
{
  "sigchainType": "Membership",
  "linkType": "ChangeMembers",
  ...
  "accountID": "19aebb...",

  "added": [
    COMMIT({
      "userID": "db2f1c...",
      "userChainSequenceNumber": 9,
    }),
    ...
  ],
  "removed": [
    COMMIT({
      "userID": "9ae3d2...",
      "userChainSequenceNumber": 4
    }),
    ...
  ]
}
\end{Verbatim}

Each link can add or remove multiple users. Each user is hidden behind a commitment so that it is
possible to prove that an individual user is part of an account without also leaking the {\userID}s
of the other members that are being added as part of the same link. Hiding the {\userID}s of the
users being removed from the account would potentially make it harder to prove that a user is indeed
still a member of a specific account without opening all the commitments. We will solve this problem
in the future by leveraging the transparency layer, but until then, clients will not rely on these
sigchains, although the server will start keeping track of them.

Links in membership sigchains do not require any signatures.

\subsection{Sigchain Fingerprints}
\label{subsec:fingerprints}
A sigchain fingerprint summarizes a user's sigchains. It is constructed as a hash of the user's user
sigchain and account sigchain tails (note that this includes the user's present and past email
addresses, as well as the current ADN due to the reference in the account sigchain). Sigchain
fingerprints are useful in two contexts: to verify new devices before approving, and to verify the
keys used to communicate with other users.

During the device addition and approval process, a malicious insider has the opportunity to conduct
a MitM attack. After the new device is added, the insider creates a copy of the
user's sigchain but replaces the keys in the final $\deviceadd{}$ link with keys controlled by the
attacker, using the same device name. The forged copy is served to older devices while the honest
sigchain is served to the new device. When the older device sees the approval interface, only the
device name is shown, and so the user may be tricked into accepting the server's device
unintentionally (which would imply sharing all previously encrypted data with the attacker).
However, in this case the two devices would necessarily display different fingerprints (as they bind
to sigchains containing different keys, and the hash is collision resistant). Users concerned about
MitM attacks can check, before approving new devices, that the fingerprint shown on
the approving device matches the one from the device(s) they intend to approve. This ensures that an
attacker cannot interfere with the process and gain access to historical encrypted data.

Sigchain fingerprints are also useful when communicating with other users. For example, when a user
emails another user for the first time, a malicious server could potentially swap out the intended
user's sigchain for an entirely fabricated sigchain with devices controlled by the server. In these
contexts, the user interface will allow users to verify the other user's sigchain fingerprint before
sending the encrypted message. This process requires the two users to securely communicate
out-of-band (for example, in-person). The sender can check that the fingerprint they see in their
client matches the recipient's fingerprint to ensure the message can be decrypted only by the
intended recipient. Similarly, the recipient can do the same for the sender to verify that the
message came from the expected sender and was not altered.

In the user interface, fingerprints are represented in hexadecimal. Users can always view their own
fingerprint from the Zoom client, and Zoom products will show the fingerprints of other users in
specific scenarios where it is useful.

\subsection{Backup Keys}
\label{subsec:backupkeys}

A backup key acts as a ``virtual device'' that can be added to a user's sigchain. Instead of
corresponding to an actual device, it is simply represented as a string of letters and digits, such
as
\[ \texttt{ZR30 4D11 5HJM RJG2 6H75 78DH B0VS 4KSF}. \] The first four characters are used as a key
identifier and are not considered private. Backup keys have at least 128 bits of private entropy, as
well as built-in error correction to tolerate small copy-pasting mistakes.

From a trusted device's Zoom client, a user can generate a backup key and add it as a new ``device''
to the user's sigchain with a $\deviceaddandapprove{}$ link. The backup key string is used to derive
a seed using sodium{}'s~\cite{libsodium} \texttt{argon2id}, which is then used to derive the
device's signing and encryption keys. Like normal devices, backup keys can be revoked.


\subsubsection{Account Recovery}

In the case that a user loses all their devices, they can log onto another device and enter their
backup key to write a $\batchapprove{}$ link to the user's sigchain that originates from the backup
key. This gives the new device access to all the old PUKs that the backup key had access to.

\subsubsection{Using Backup Keys for Escrow}\label{subsec:backupkeyescrow}

\textbf{Note:} Escrow using backup keys is not currently available. Users can still
manually generate backup keys and send them to their IT administrator out of band.

Organizations using Zoom may wish to have access to E2E communications involving their members, to
protect against data loss or to support legal retention and discovery. We refer to solutions that
give such organizations cryptographic access as \textit{escrow}.  While we are building a more
usable and robust escrow solution, backup keys can initially provide the minimal necessary
functionality: users can generate backup keys and send them to their IT department for safekeeping.
When escrow is enabled for an account, users in that account are notified before they are allowed to
continue use of Zoom products. Each user receives notice that their data is being escrowed the
first time they access their Zoom Mail Service email on one of their devices. The user should verify
the escrow administrator's sigchain fingerprint before continuing, after which the user's client
automatically sends an end-to-end encrypted email with a backup key to their account administrator.
